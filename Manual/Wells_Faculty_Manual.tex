\documentclass{manual}
\usepackage[colorlinks=true, urlcolor=red, bookmarksnumbered=true]{hyperref}
\usepackage{enumitem} %http://www.tex-tipografia.com/archive/enumitem.pdf
\usepackage{draftwatermark} %http://www.math.washington.edu/tex-archive/macros/latex/contrib/draftwatermark/draftwatermark.pdf
\SetWatermarkScale{6.0}
%\usepackage{enumerate} %http://texblog.org/2008/10/16/lists-enumerate-itemize-description-and-how-to-change-them/
%\setenumerate{labelindent=\leftskip, leftmargin=*, align=left}
\setenumerate{labelindent=1.5em, leftmargin=*, align=left}

%\usepackage[cm]{fullpage}
\usepackage{fullpage} %http://en.wikibooks.org/wiki/LaTeX/Page_Layout
\usepackage{parskip} %http://en.wikibooks.org/wiki/LaTeX/Paragraph_Formatting
\usepackage{fancyhdr} %http://www.latex-community.org/forum/viewtopic.php?f=47&t=3512
\renewcommand{\headsep}{1em} %Give space from top line

%\usepackage{endnotes} %http://tex.stackexchange.com/questions/56145/is-there-a-way-to-move-all-footnotes-to-the-end-of-the-document
%\let\footnote=\endnote

\usepackage{cleveref} %http://www.dr-qubit.org/latex/cleveref.pdf
\crefname{section}{ARTICLE}{ARTICLES}
\crefname{subsection}{Section}{Sections}
\crefname{subsubsection}{Subsection}{Subsections}
\crefname{paragraph}{Paragraph}{Paragraphs}
\addtolength{\headheight}{\baselineskip}
%\newcommand{\modified}[1]{\marginpar{[#1]}}
%\newcommand{\modified}[1]{\footnote{Modified: [#1]}}
\newcommand{\modified}[1]{}
%\newcommand{\oldbreak}[1]{\vfill((#1))\newpage}
%\newcommand{\oldbreak}[1]{\marginpar{((#1))}}
\newcommand{\oldbreak}[1]{}

\addtocounter{secnumdepth}{1} % LaTeX companion 2.2.1 p 24
\addtocounter{tocdepth}{-1} % Article and Section only
%\addtocounter{tocdepth}{1} %Article, Section, Subsection, Paragraph
\let\stdsection\section %http://blog.dreasgrech.com/2010/01/starting-new-page-with-every-section-in.html
\renewcommand\section{\newpage\stdsection}

\usepackage{color} %http://en.wikibooks.org/wiki/LaTeX/Colors
\newcommand{\keyword}[1]{\textcolor{black}{#1}}
\newcommand{\facman}{\keyword{\underline{Faculty Manual}}~}
\newcommand{\constitution}{Wells College Collegiate Bylaws and Constitution~}
\newcommand{\exoff}{\keyword{\textit{ex officio}}~}
\newcommand{\adho}{\keyword{\textit{ad hoc}}~}

%http://www.latex-community.org/forum/viewtopic.php?f=4&t=251
\let\oldsection\section
\renewcommand\section{\leftskip=0em\oldsection}

\let\oldsubsection\subsection
\renewcommand\subsection{\leftskip=0em\oldsubsection}

\let\oldsubsubsection\subsubsection
\renewcommand\subsubsection{\leftskip=0.5em\oldsubsubsection}

\let\oldparagraph\paragraph
\renewcommand\paragraph{\leftskip=1em\oldparagraph}

\newcommand{\SUBparagraph}[1]{\textit{SUB-PARAGRAPH: #1}}

%Edits

\newcommand{\they}{(pronoun)~ }
\newcommand{\They}{(Pronoun)~ }
\newcommand{\their}{(pronoun)~ }
\newcommand{\Their}{(Pronoun)~ }
\newcommand{\them}{(pronoun)~ }
\newcommand{\themself}{(pronoun)~ }

%Symbol levels
\newcommand{\itemLevelA}{\alph*.}
\newcommand{\itemLevelB}{\arabic*)}
\newcommand{\itemLevelC}{\alph*)}
\newcommand{\itemRefA}{\alph*}
\newcommand{\itemRefB}{\arabic*}
\newcommand{\itemRefC}{\alph*}

\begin{document}

\title{FACULTY MANUAL (Draft)}
\author{Wells College\\Aurora, New York}

\maketitle
  \vfill
  \begin{center}
  Revised [Revisions in progress]
  \end{center}

\newpage

\pagestyle{fancy}
\fancyhead[R]{\thepage \addtocounter{articlePage}{1}}
\fancyhead[L]{Wells College Faculty Manual}

\tableofcontents
\newpage

\fancyfoot[C]{\thesection-\thearticlePage}

%% ARTICLE I. FACULTY AND GOVERNANCE
\section{FACULTY AND GOVERNANCE}\label{art:FacultyAndGovernance}
\begin{center}(As ordered by the Board of Trustees)\end{center}

%_% Section 1. Personnel
\subsection{Personnel}\label{sec:Personnel}

%__% A. Faculty
\subsubsection{Faculty}\label{sub:Faculty}

%___% 1. Definition
\paragraph{Definition}
The Faculty shall consist of the officers of instruction holding the ranks of Professor, Associate Professor, Assistant Professor, and Lecturer, and the President of the College, and the Provost and Dean of the College.

%___% 2. Responsibilities
\paragraph{Responsibilities}\label{sub:Responsibilities}
\begin{enumerate}[label=\itemLevelA,ref=\itemRefA]
%____% a. 
\item The Faculty shall have primary responsibility for teaching, curriculum development, research, and other scholarly activities.
%____% b. 
\item The Faculty shall prescribe (subject to the approval of the Board of Trustees on significant matters) requirements for admission, courses of study, conditions of graduation, rules and methods for the conduct of the educational work of the college, and shall recommend to the Board candidates for degrees.
%____% c. 
\item Through the appropriate committees, the Faculty shall advise the relevant Administrator(s) on such matters as admissions, scholarships, academic computing and other technologies, budget, personnel policies, and the quality of student life.
%____% d. 
\item The Faculty shall also have oversight over those aspects of student life related to the educational process. 
\end{enumerate}

%__% B. Provost and Dean of the College; Associate Provost of the College
\subsubsection{Provost and Dean of the College; Associate Provost of the College}

%___% 1. Appointment and rank
\paragraph{Appointment and rank}
The Provost and Dean of the College (hereafter referred to as the Dean, or the Dean of the College) shall be appointed by the President in consultation with the Advisory Committee and shall have professorial rank.

%___% 2. Responsibilities
\paragraph{Responsibilities}

\begin{enumerate}[label=\itemLevelA,ref=\itemRefA]
%____% a. 
\item The Dean of the College shall be responsible for the academic life of the College and shall have the authority to deal with all matters of an academic nature that do not require Faculty approval.
%____% b. 
\item The Dean shall, at all times, keep the President informed of important issues and decisions and shall advise the President on academic matters when action by the President is required.
%____% c. 
\item In consultation with the President and the Chief Operating Officer, the Dean shall prepare the academic budget of the College.
%____% d. 
\item The Dean shall regularly convene the Dean's Council (see \cref{art:FacultyAndGovernance}, \cref{sec:DivisionsAndMajorAndMinorPrograms}.\labelcref{sub:Divisions}.\labelcref{par:DivisionChairs}.\labelcref{item:responsibilities01}.\labelcref{iitem:deansCouncil02}), be ultimately responsible for academic advice to students, perform all other duties specifically required by the \facman, and shall assume the academic responsibilities of the President in the prolonged absence or inability of the President to act. The Dean shall be a member of the Educational Policy Committee and shall chair the Curriculum Committee.
%____% e. 
\item The Dean of the College and the Dean's Council shall regularly report to the Board of Trustees, through the Academic Affairs Committee of the Board of Trustees, on the size and distribution of the Faculty (see \cref{art:RegulationsConcerningFacultyStatus} \cref{sec:TeachingFaculty}.\labelcref{sub:Core}), noting any increases, decreases, or reallocations.
\end{enumerate}

%___% 3. Associate Provost of the College
\paragraph{Associate Provost of the College}
\begin{enumerate}[label=\itemLevelA,ref=\itemRefA]
%____% a. 
\item The Dean of the College may,  in consultation with the Advisory Committee, appoint an Associate Provost of the College, who shall assist in the execution of the duties and responsibilities of the Dean.
%____% b. 
\item The Associate Provost may act as a proxy for the Dean at meetings in which the Dean is a voting member.
\end{enumerate}

%_% Section 2. Faculty Meetings
\subsection{Faculty Meetings} \label{sec:FacultyMeetings}
%__% A. Dates, Times, and Types of Meetings
\subsubsection{Dates, Times, and Types of Meetings}

%___% 1. 
\paragraph{Regular meetings} 
The Faculty shall normally meet on the second Tuesday afternoon of each month of the academic year. 

%___% 2. Special meetings
\paragraph{Special meetings}
\begin{enumerate}[label=\itemLevelA,ref=\itemRefA]
%____% a. 
\item The President or the Dean of the College may call special meetings.
%____% b. 
\item In addition, special meetings shall be called by the Secretary upon the request of three members of the Faculty, whose names shall be included in the official minutes.
\end{enumerate}

%__% B. Agenda
\subsubsection{Notice of Agenda}

%___% 1. 
\paragraph{Regular Business} 
No business may be transacted at a meeting of the Faculty unless it has been placed on the agenda by 1:00 p.m. of the academic day before the meeting, at which time the agenda shall be made available to the Faculty.

%___% 2. 
\paragraph{Emergency Business} 
With the approval of the presiding officer and the Secretary of the Faculty, exception may be made in the case of an emergency, whereupon the agenda shall be made available to members of the Faculty as soon as the meeting is called.

%__% C. Procedure
\subsubsection{Procedure} \label{sub:Procedure}
%___% 1. Presiding Officer
\paragraph{Presiding Officer}
Ordinarily, the President shall be the presiding officer at Faculty meetings; in the President's absence, the Dean of the College shall preside.
%___% 2. Secretary
\paragraph{Secretary}
A secretary, elected from the Faculty, shall keep a record of the proceedings and distribute minutes to the Faculty.
%___% 3. Quorum
\paragraph{Quorum}
In calculating a quorum, only full-time Faculty members shall be counted. 
%___% 4. Voting
\paragraph{Voting} \label{par:Voting}

\begin{enumerate}[label=\itemLevelA,ref=\itemRefA]
%____% a. 
\item \label{item:voiceAnVote03} Voice and Vote
\begin{enumerate}[label=\itemLevelB,ref=\itemRefB]
%_____% 1) 
\item All Faculty members as designated in \cref{sec:Personnel}.\labelcref{sub:Faculty} shall have voice and vote in Faculty meetings, except that part-time Faculty shall not vote during their first year of service.
%_____% 2) 
\item \label{iitem:admins04} The following shall also have voice and vote in faculty meetings: the Dean of Students, the Registrar, and the Director of the Library.
\end{enumerate}

%____% b. 
\item Voice But No Vote
\begin{enumerate}[label=\itemLevelB,ref=\itemRefB]
%_____% 1) 
\item Other special appointment Faculty shall have voice but no vote.
%_____% 2) 
\item The Advisory Committee, in consultation with the President of the College, shall designate those administrators and staff members, not otherwise listed in \cref{sec:FacultyMeetings}.\labelcref{sub:Procedure}.\labelcref{par:Voting}.\labelcref{item:voiceAnVote03}.\labelcref{iitem:admins04}, who shall have voice but no vote in the Faculty meetings. The Secretary of the Faculty shall maintain this list and make it available to the Faculty.

At the time of passage of this Article, the following administrators and staff members had voice but no vote in Faculty Meetings: 
Assistant to the President; 
Associate Director of Admissions; 
Associate Provost; 
Chief Operating Officer; 
Dean of Students; 
Director of Alumnae Relations; 
Director of Communications and Marketing; 
Director of Experiential Learning and Career Services; 
Director of Financial Aid; 
Director of International Programs; 
Librarians for Public Services and Outreach, Technical Services and System Management, and Circulation and Reserves; 
Vice President for Advancement. 
\end{enumerate}
\end{enumerate}

%___% 5. Students
\paragraph{Students}
\begin{enumerate}[label=\itemLevelA,ref=\itemRefA]
%____% a. 
\item One student representative from each of the following shall have voice but no vote in Faculty meetings: the Educational Policy Committee, the Curriculum Committee, and the Collegiate Cabinet.
%____% b. 
\item Students shall not be present for discussion or action on agenda items concerned with individual students, recommendations for degrees or prizes, or for any items designated as confidential by the presiding officer. Students shall not receive minutes of meetings that contain these items.
\end{enumerate}

%___% 6. Robert's Rules of Order
\paragraph{Rules of Order}
At its meetings, the Faculty shall be governed by Robert's \emph{Rules of Order}. 

%__% D. Minutes
\subsubsection{Minutes}
%___% 1. 
\paragraph{Distribution} 
Minutes shall be distributed no later than 24 hours prior to the next regular meeting to members of the Faculty and to those persons designated above as having voice at Faculty meetings.

%___% 2. 
\paragraph{Archival} 
The official minutes shall be initialed by the Secretary of the Faculty and deposited in the Registrar's office.

%Section 3. Divisions, and Major and Minor Programs
\subsection{Divisions, and Major and Minor Programs}\label{sec:DivisionsAndMajorAndMinorPrograms}
%__% A. Divisions 
\subsubsection{Divisions}\label{sub:Divisions}
%___% 1. Structure
\paragraph{Structure}
\begin{enumerate}[label=\itemLevelA,ref=\itemRefA]
%____% a. 
\item The academic program shall be organized into the following divisions: Arts, Humanities, Natural and Mathematical Sciences, Social Sciences. All majors, minors, and other teaching areas shall be incorporated into one of the existing divisions.
%____% b. 
\item Each Faculty member shall be assigned to one division, and shall have voting privileges in that division only.
\end{enumerate}

%___% 2. Responsibilities of Division Members
\paragraph{Responsibilities of Division Members}
\begin{enumerate}[label=\itemLevelA,ref=\itemRefA]
%___% a. 
\item Each member of a division is responsible for active, regular, and collegial participation in the affairs of the division.
%___% b. 
\item Each member shall attend meetings, have voice and vote as per \cref{sec:FacultyMeetings}.\labelcref{sub:Procedure}.\labelcref{par:Voting} above in the formulation of divisional policy, and perform appropriate divisional duties.
\end{enumerate}

%___% 3. Division Chairs
\paragraph{Division Chairs}\label{par:DivisionChairs}
\begin{enumerate}[label=\itemLevelA,ref=\itemRefA]
%____% a. 
\item Election
\begin{enumerate}[label=\itemLevelB,ref=\itemRefB]
%_____% 1) 
\item The voting members of each division, as per \cref{sec:FacultyMeetings}.\labelcref{sub:Procedure}.\labelcref{par:Voting} above, shall elect a chairperson for a term of two academic years with the possibility of re-election for a second consecutive two-year term. Willingness to stand for election as division chair shall be taken as indicating willingness to serve the entire term. No faculty member may serve again as division chair until one year shall have elapsed since the end of the previous term of service.
%_____% 2) 
\item The results of such elections shall be conveyed to the Committee on Committees no later than two weeks prior to the close of classes in the Spring Semester.
%_____% 3) 
\item In the event that no one has been elected to chair a division, the Dean of the College, in consultation with the Advisory Committee on Faculty Personnel, shall fill the vacancy, preferably by appointment of a member of the division.
%_____% 4) 
\item Faculty members holding administrative appointments shall not be eligible to serve as division chairs.
%_____% 5)
\item  Because division chairs serve on the Evaluation Committee, they must be tenured.
\end{enumerate}

%____% b. 
\item Vacancy
\begin{enumerate}[label=\itemLevelB,ref=\itemRefB]
%_____% 1) 
\item In the event of a vacancy during the term of a division chair, a new election shall be held when the vacancy occurs, with voting rights as outlined in \cref{sec:FacultyMeetings}.\labelcref{sub:Procedure}.\labelcref{par:Voting} above.
%_____% 2) 
\item The term shall expire in June of the academic year following the one in which the vacancy occurs.
%_____% 3) 
\item The division chair so elected shall be eligible for re-election to a two-year term. 
\end{enumerate}

%____% c. 
\item \label{item:responsibilities01} Responsibilities
\begin{enumerate}[label=\itemLevelB,ref=\itemRefB]
%_____% 1) 
\item Administration

The chairperson shall be responsible for
\begin{enumerate}[label=\itemLevelC,ref=\itemRefC]
%______% a) 
\item the general administration of the division by democratic methods.
%______% b) 
\item communicating concerns and recommendations of the division.
%______% c) 
\item supervising the preparation of the annual divisional budget and for its observance after approval. Does this really happen?
%______% d) 
\item general oversight of the students and procedures of instruction in the division.
\end{enumerate}

%_____% 2) 
\item Division Meetings
\begin{enumerate}[label=\itemLevelC,ref=\itemRefC]
%______% a) 
\item The division chair shall call a meeting of the division at least twice each semester, whenever the chair shall consider it desirable, or whenever requested to do so by three members of the division.
%______% b) 
\item Whenever appropriate, members of other divisions shall be invited to meetings.
%______% c) 
\item Voting rights shall be consistent with procedures in \cref{sec:FacultyMeetings}.\labelcref{sub:Procedure}.\labelcref{par:Voting} above.
\end{enumerate}

%_____% 3) 
\item Committees
%______% a) 
\item The division chairs shall serve as faculty representatives on the Curriculum Committee and shall also serve on Evaluation Committees. 

%_____% 4) 
\item \label{iitem:deansCouncil02} Dean's Council
\begin{enumerate}[label=\itemLevelC,ref=\itemRefC]
%______% a) 
\item The division chairs shall be convened regularly by the Dean as a Dean's Council.

%______% b) 
\item The Council shall act as an advisory group to the Dean, and as the principal channel of communication between the Dean and the members of the divisions on matters affecting the academic and library budgets, including the allocation of positions, and on other administrative, library, and student matters not requiring Faculty approval.
%______% c) 
\item Each division chair shall report to their division on all matters discussed at the Dean's Council that affect the work of the division and shall seek the advice of the division on matters of policy and decision-making affecting the division.
\end{enumerate}

\end{enumerate}

\end{enumerate}

%__% B. Major Programs 
\subsubsection{Major Programs}
%___% 1. Structure
\paragraph{Structure}
\begin{enumerate}[label=\itemLevelA,ref=\itemRefA]
%____% a. 
\item Each major program consists of faculty members who shall be responsible for offering courses and administering a particular major listed in the Wells College Catalog.
%____% b. 
\item Major programs shall have purview over the concentrations in the particular major with its minors, as well as other related minors and relevant courses.
%____% c. 
\item The faculty of each major program shall meet regularly during the semester, as needed, to plan curricular offerings, prepare a budget, and coordinate other business of the major and any minor within that major.
%____% d. 
\item Voting rights for members of the major shall be consistent with procedures in \cref{sec:FacultyMeetings}.\labelcref{sub:Procedure}.\labelcref{par:Voting} above. 
\end{enumerate}

%___% 2. Responsibilities of Major Program Members
\paragraph{Responsibilities of Major Program Members}
\begin{enumerate}[label=\itemLevelA,ref=\itemRefA]
%____% a. 
\item Members of the major program shall keep clear records of courses, and shall plan and teach courses in such a way as to further the best interests of the major, the division, and the College.
%____% b. 
\item Each member of a major program shall be responsible for active and regular participation in the affairs of that major.
%____% c. 
\item Each member shall attend meetings, have voice and vote as per \cref{sec:FacultyMeetings}.\labelcref{sub:Procedure}.\labelcref{par:Voting} above in the formulation of policy, and perform appropriate major program duties.
\end{enumerate}

%___% 3. Chairs of the Major Programs
\paragraph{Chairs of the Major Programs}
\begin{enumerate}[label=\itemLevelA,ref=\itemRefA]
%____% a. 
\item Election

The faculty of each major program shall elect a chair for a period of two years who may be re-elected any number of times. In the case of major programs with only one full-time faculty member, that faculty member shall act as chair.
%____% b. 
\item Responsibilities

The chair shall be responsible for the general administration of business pertaining to the major program.
\end{enumerate}

%___% 4. Recommendations of the Major
\paragraph{Recommendations of the Major}
\begin{enumerate}[label=\itemLevelA,ref=\itemRefA]
%____% a. 
\item Curricular and other recommendations of the major shall be brought to the division for consideration and appropriate action.
\end{enumerate}

%__% C. Minor Programs that are not in a Major Program 
\subsubsection{Minor Programs that are not in a Major Program}
%___% 1. Structure
\paragraph{Structure}
\begin{enumerate}[label=\itemLevelA,ref=\itemRefA]
%____% a. 
\item Minor programs that do not belong to a major program shall exist within one division.

%____% b. 
\item The faculty of each minor field shall meet at least once a year to evaluate and plan curricular offerings. 
%____% c. 
\item Voting rights for members of the minor shall be consistent with procedures in \cref{sec:FacultyMeetings}.\labelcref{sub:Procedure}.\labelcref{par:Voting} above.
\end{enumerate}

%___% 2. Responsibilities of Minor Program Members
\paragraph{Responsibilities of Minor Program Members}
\begin{enumerate}[label=\itemLevelA,ref=\itemRefA]
%____% a. 
\item Members of the minor program shall keep clear records of courses, and plan and teach courses in such a way as to further the best interests of the minor, the related majors and divisions, and the College.
%____% b. 
\item Each member of a minor program is responsible for active and regular participation in the affairs of that minor.
%____% c. 
\item Each member shall attend meetings, have voice and vote as per \cref{sec:FacultyMeetings}.\labelcref{sub:Procedure}.\labelcref{par:Voting} above in the formulation of policy, and perform appropriate minor program duties including advising.
\end{enumerate}

%___% 3. Coordinators of the Minor Programs
\paragraph{Coordinators of the Minor Programs}
\begin{enumerate}[label=\itemLevelA,ref=\itemRefA]
%____% a. 
\item The faculty of each minor program shall elect a coordinator for a period of two years with the possibility of re-election. In the case of minor programs with only one faculty member, that faculty member shall act as coordinator.
%____% b. 
\item The coordinator shall be responsible for the general administration of business pertaining to the minor program, especially for advising about the minor and the signing of Declaration of Minor forms for students in the minor.
\end{enumerate}

%_% Section 4. Committee Service
\subsection{Committee Service}\label{sec:CommitteeService}
%__% A. General information
\subsubsection{General information}\label{sub:GeneralInformation}

\paragraph{Primacy and Service Expectation}

\begin{enumerate}[label=\itemLevelA,ref=\itemRefA]
%___% 1. 
\item The work of the Faculty is done through its standing and \adho committees.

%___% 2. 
\item Regular committee service is expected of all full-time teaching faculty.
\end{enumerate}

%___% 3. 
\paragraph{Outside Experts}
If the Administration or Faculty believes that there is a need for the services of outside professional experts in areas described in \cref{art:FacultyAndGovernance} \cref{sec:Personnel}.\labelcref{sub:Faculty}.\labelcref{sub:Responsibilities}, the appropriate Faculty committees or the Advisory Committee, in the spirit of shared governance, shall be consulted prior to the decision to engage such experts.

%__% B. Criteria for Service
\subsubsection{Criteria for Service}
\paragraph{Eligibility and Service}

\begin{enumerate}[label=\itemLevelA,ref=\itemRefA]
%___% 1. 
\item Faculty members of standing committees shall be those who are teaching at least half-time.

%___% 2. 
\item Unless otherwise specified, terms of office on committees shall be three years. Terms shall be staggered such that there is an annual rotation within the committee.

%___% 3. 
\item No faculty member may serve again on a given committee until one year shall have elapsed since the previous full term.

%___% 4. 
\item Ordinarily, no faculty member shall serve on more than one elected committee.
\end{enumerate}

%___% 5. 
\paragraph{Exclusivity}
Under no circumstances shall any faculty member serve simultaneously on two of the following committees: the Advisory Committee on Faculty Personnel, the Faculty Evaluation Committee, or the Review Committee. In addition, no faculty member shall serve simultaneously on both the Curriculum Committee and the Educational Policy Committee.

%__% C. President and Dean
\subsubsection{President and Dean}
%___% 1. 
\paragraph{President}
The President shall be a member \exoff of all standing faculty committees except the Advisory Committee on Faculty Personnel, the Committee on Committees, the Evaluation Committee and the Review Committee.

%___% 2. 
\paragraph{Dean}
The Dean of the College shall chair the Curriculum Committee, shall be a member of the Educational Policy Committee and shall be a member \exoff of all faculty committees except the Advisory Committee on Faculty Personnel, the Committee on Committees, the Evaluation Committee and the Review Committee.

%__% D. Students
\subsubsection{Students}
Student members of Faculty Committees shall be elected by the student body. Unless otherwise stated, student members shall be voting members of the committees.

%__% E. Procedure
\subsubsection{Procedure}
%___% 1. 
\paragraph{Activity}
Every committee shall hold at least one meeting per semester.

\paragraph{Leadership}

\begin{enumerate}[label=\itemLevelA,ref=\itemRefA]
%___% 2. 
\item Unless otherwise specified, each committee shall elect a chairperson annually. When there is no chairperson, the senior-ranking faculty member on the committee shall convene the committee.

%___% 3. 
\item At the beginning of the fall semester, the chairperson shall be responsible for the establishment by the committee of its mandate for the year based upon the directives given in the \facman. At the end of the spring semester, the chairperson shall outline unfinished business with recommendations for the coming year to be handed on to the new chairperson.

%___% 4. 
\item Each committee shall designate a secretary, who shall keep minutes of all committee meetings.
\end{enumerate}

\paragraph{Reports}

%___% 5. 
\begin{enumerate}[label=\itemLevelA,ref=\itemRefA]
\item Each committee shall report to the Faculty at least once in each semester of the academic year. Normally, the chairperson shall deliver the committee report.

%___% 6. 
\item The Faculty may at any time request a committee report. 
\end{enumerate}

%__% F. Sub-Committees
\subsubsection{Sub-Committees}
%___% 1. 
\paragraph{Formation}
All standing committees shall be empowered to form sub-committees to further specific aims. 

\paragraph{Membership}
\begin{enumerate}[label=\itemLevelA,ref=\itemRefA]
%___% 2. 
\item At least one member of the standing committee shall be a member of the sub-committee.

%___% 3. 
\item The chairperson of the standing committee shall appoint all sub-committees, with the committee's approval.
\end{enumerate}

%__% G. Ad Hoc Committees
\subsubsection{Ad Hoc Committees}

\paragraph{Formation}
\begin{enumerate}[label=\itemLevelA,ref=\itemRefA]
%___% 1. 
\item The Faculty, the President or the Dean of the College may create \adho committees. The work of such committees shall not usurp the work of standing committees.

%___% 2. 
\item If the Faculty creates an \adho committee, the President and Dean shall be informed promptly in writing, defining the function of the committee.

%___% 3. 
\item If the President or Dean creates an \adho committee, the Faculty shall be informed promptly in writing, defining the function of the committee.
\end{enumerate}

%___% 4. 
\paragraph{Franchise}
Recommendations of \adho committees shall not be implemented without the full consideration and approval of the Faculty.

%___% H. Elections
\subsubsection{Elections}\label{sub:Elections}
%___% 1. General Regulations
\paragraph{General Regulations}\label{par:GeneralRegulations}
\begin{enumerate}[label=\itemLevelA,ref=\itemRefA]

%____% a. 
\item All officers of instruction holding the rank of Professor, Associate Professor, Assistant Professor,  or Lecturer teaching at least half-time (as determined by the Dean of the College) shall be considered eligible for committee service, provided that they have taught at the College for at least one year and that one year has elapsed since the previous full term of service on that committee.
%____% b. 
\item Only officers of instruction teaching at least half-time may vote in committee elections, except that Lecturers shall not vote in their first year of service. Faculty on leave may vote in committee elections.
%____% c. 
\item Nomination ballots may be sent electronically or via mail. Normally all election voting shall be by mail. Any paper ballots shall be sent out from and returned to the Office of the Secretary of the Faculty, where they shall be deposited until such time as the Secretary of the Faculty and one designated member of the Committee on Committees shall tally them, and thereafter until the end of the following year.
%____% d. 
\item \label{item:ballots05} Ballots, for both nomination and election, shall list in alphabetical order for each committee the names of those persons remaining on the committee, the number and terms of vacancies, the names of those eligible or nominated, and the date by which ballots must be returned.
%____% e. 
\item The Secretary of the Faculty and the following committees shall be elected directly by the Faculty: Educational Policy Committee, Academic Standing and Advising, Admissions and Financial Aid, Advisory Committee on Faculty Personnel, Committee on Committees, and Review Committee. Positions on Community Court, Judicial Appeals Board, Arts and Lectures Committee, Safety and Security Committee are appointed.
%____% f. 
\item A temporary vacancy caused by the absence or leave for one or two semesters of an elected member of any committee shall be filled for no less than one academic year by a duly elected substitute. Other vacancies shall be filled until the ensuing elections through reversion to the previous ballot.
\end{enumerate}

%___% 2. 
\paragraph{Voting Procedures and Schedule}
\begin{enumerate}[label=\itemLevelA,ref=\itemRefA]
%____% a. 
\item Results of election of division chairs shall be conveyed to the Committee on Committees no later than two weeks prior to the close of classes in the Spring Semester.
%____% b. 
\item At least two weeks before the close of classes in the Spring Semester, the Committee on Committees shall initiate the process described in \cref{sub:Elections}.\labelcref{par:GeneralRegulations}.\labelcref{item:ballots05}. above. Any faculty members wishing to withhold their name from nomination must inform the Secretary of the Faculty in writing.
%____% c. 
\item At the time of nominations, the Secretary of the Faculty shall call the attention of the Faculty to the desirability of having the various disciplines and a balance of experience and inexperience represented on the committees.
%____% d. 
\item On the nomination ballot, for each committee with a vacancy,  faculty members may indicate their willingness to serve, and may propose as many different candidates for each committee as there are vacancies. The names of the willing candidates receiving the highest number of nominations shall appear on the election ballot; the final number of nominations for each committee shall be two more than there are vacancies. In the case of a tie in the last place, all the names of those faculty members willing to serve who are involved in the tie shall appear on the election ballot.
%____% e. 
\item Immediately after tallying the nominations, or not more than 10 days after the nomination ballot was sent out, the Committee shall mail to each member of the voting faculty an election ballot, which is to be returned within three days.
%____% f. 
\item On the election ballot each faculty member may vote for as many different persons as there are vacancies, each vote having the value of one. Nominees receiving the largest number of votes shall be elected. Any ties shall be settled by the Committee on Committees.
%____% g. 
\item The Committee on Committees shall also use nominating ballots as the basis for appointments to Community Court, Judicial Appeals Board, Lectures and Concert Committee, and the Safety and Security Committee.
%____% h. 
\item Not more than ten days after mailing of the election ballot, the Committee shall announce in writing the results of the election and the resulting composition of the standing committees.
%____% i. 
\item The provisions above do not apply to the \exoff members of committees. 
\end{enumerate}

%___% I. 
\subsubsection{Confidentiality}

%___% 1. 
\paragraph{Protected matters}
Committees shall maintain confidentiality in dealing with personnel evaluations and on other matters where protection of individual privacy is paramount.

%___% 2. 
\paragraph{Unprotected matters}
All other matters shall be considered to be non-confidential.

%___% 3. 
\paragraph{Reporting}
In their annual reports, each committee shall report the number of occasions in which confidentiality was voted, and the general reasons for its adoption.

%Section 5. Faculty Committees

\subsection{Faculty Committees} \label{sec:FacultyCommittees}
Unless otherwise stated, standing committees shall be elective.

%__% A. 
For General Information and Criteria for Service, see \cref{sec:CommitteeService}.\labelcref{sub:GeneralInformation} above.

%__% B. Standing Committees
\subsubsection{Standing Committees}\label{sub:StandingCommittees}

%___% 1. Academic Standing and Advising
\paragraph{Academic Standing and Advising}

\begin{enumerate}[label=\itemLevelA,ref=\itemRefA]
%____% a. 
\item Membership
\begin{enumerate}[label=\itemLevelB,ref=\itemRefB]
%_____% 1) 
\item The Academic Standing and Advising Committee shall consist of three elected members of the Faculty, one of whom shall be chairperson; the Dean of Students; the Registrar; and the Director of Academic Advising.
%_____% 2) 
\item The three elected faculty members shall also serve as representatives to the Student-Faculty-Administration Board.
\end{enumerate}
%____% b. 
\item Responsibilities

The Committee shall oversee the application of academic regulations regarding standing, deficient students, re-admission, requirements for the undergraduate degree, academic honors and prizes, student advising, and shall propose new policy as necessary.

\end{enumerate}

%___% 2. Admissions and Financial Aid 
\paragraph{Admissions and Financial Aid}

\begin{enumerate}[label=\itemLevelA,ref=\itemRefA]
%____% a. 
\item Membership
\begin{enumerate}[label=\itemLevelB,ref=\itemRefB]
%_____% 1) 
\item The Admissions and Financial Aid Committee shall consist of three members of the Faculty, one of whom shall be chairperson, the Dean of Students, the Director of Admissions and Financial Aid, and the Director of Financial Aid.
%_____% 2) 
\item The Director of Athletics and two students shall serve as non-voting advisors.
\end{enumerate}
%____% b. 
\item Responsibilities
\begin{enumerate}[label=\itemLevelB,ref=\itemRefB]
%_____% 1) 
\item The Committee shall have oversight on matters involving recruitment, admissions, and retention, and shall consider the curricular and financial implications of admissions and financial aid.
%_____% 2) 
\item In fulfilling its charge, the Committee shall review and assess areas such as admission standards, goals and methods of recruitment and retention, the effectiveness of programs for recruitment and retention (such as campus visits, orientation, merit awards, experiential learning support, intercollegiate athletics, and financial aid packages), and curricular and financial implications of all policies. The Committee shall also review admissions literature and Faculty involvement in the admissions process.
%_____% 3) 
\item The Committee shall bring recommendations about requirements and standards for admissions to the Faculty for approval.
%_____% 4) 
\item The Committee shall participate in the awarding of academic merit scholarships and make decisions on marginal applications.

\end{enumerate}
\end{enumerate}

%___% 3. Advisory Committee on Faculty Personnel
\paragraph{Advisory Committee on Faculty Personnel}\label{par:AdvisoryCommitteeOnFacultyPersonnel}

\begin{enumerate}[label=\itemLevelA,ref=\itemRefA]
%____% a. 
\item Membership \label{item:membership06}
\begin{enumerate}[label=\itemLevelB,ref=\itemRefB]
%_____% 1) 
\item The Advisory Committee on Faculty Personnel shall consist of four tenured voting members of the Faculty who hold no administrative office, and who have been at Wells College at whatever rank for at least three years.
%_____% 2) 
\item At least one of the members shall be a woman and at least one a man.
%_____% 3) 
\item\label{iitem:crossdiscipline07} No more than one member of the same discipline as listed in the Catalog (e.g., Professor of Mathematics and Computer Science, or Assistant Professor of History) shall serve simultaneously.
%_____% 4) 
\item No faculty member shall serve simultaneously on the Advisory Committee and either the Evaluation Committee or the Review Committee.
\end{enumerate}
%____% b. 
\item Responsibilities
\begin{enumerate}[label=\itemLevelB,ref=\itemRefB]
%_____% 1) 
\item The Committee shall represent the Faculty in defining, recommending and reviewing the policies of the College on all matters covered in \cref{art:RegulationsConcerningFacultyStatus} of the Manual, as well as on fringe benefits and other forms of compensation for the Faculty, and it shall see that the policies are administered equitably.
%_____% 2) 
\item The Committee shall meet informally with the Faculty at the beginning of each semester to discuss matters of general concern on which the Committee plans to deliberate during that semester. Recommendations made by the Committee on matters of general concern to the Faculty shall have the approval of the Faculty.
%_____% 3) 
\item Any member of the Faculty may request a hearing before the Committee at any time on any matter rightfully the concern of the Committee.
%_____% 4)
\item The members of the Committee shall, whenever practicable, meet with candidates for appointment to the Faculty.

%_____% 5) 
\item It shall review annually the rank and status, with regard to tenure or promotion, of all members of the Faculty.
%_____% 6) 
\item The Committee shall, before giving approval to any appointment, reappointment, promotion, or granting of tenure, review the status of all other faculty members in the rank(s) involved, in order to assure that serious consideration be given at all times to the matter of equity in appointments, promotions, and tenure.
%_____% 7) 
\item The Committee shall receive reports from the Evaluation Committee concerning candidates for re-appointment, tenure or promotion. The Advisory Committee shall convey to the Dean of the College in writing the results of its review of those candidates. See \cref{sec:FacultyCommittees}.\labelcref{sub:StandingCommittees}.\labelcref{par:FacultyEvaluationCommittee}.\labelcref{item:procedure08}.\labelcref{item:results09} below.
%_____% 8) 
\item When there is disagreement between the Dean of the College or the President and the Committee on any matter requiring the Committee's approval, the Committee shall communicate its views to the Board of Trustees and, if it so requests, be granted a hearing. This provision shall not be taken as limiting the right of any member of the Committee to submit to the President a minority report on any recommendation of the Committee and to have that report submitted to the Board of Trustees.
%_____% 9) 
\item The Committee shall foster faculty research by recommending and maintaining a policy for the distribution by Wells College of funds for research and other professional purposes, and by acting on specific requests for research grants. These funds are allocated to the Advisory Committee by the Dean of the College.
%_____% 10) 
\item One member of the Committee shall serve as one of the Faculty representatives to the Academic Affairs Committee of the Board of Trustees. This member shall have full voice, and shall vote in all matters exclusive of decisions regarding reappointment, promotion, and tenure.
%_____% 11) 
\item One member shall serve as budget representative to the Administration.
%_____% 12) 
\item The Committee shall review Tenured Faculty at least once every seven years as provided by \cref{art:RegulationsConcerningFacultyStatus} \cref{sec:AppointmentReappointmentPromotionAndTenure}.\labelcref{sub:TenuredAndTenureTrackFaculty}.\labelcref{par:ReviewOfTenuredFaculty} of this Manual.
%_____% 13) 
\item The Committee shall review part-time, visiting and other non-tenure-track officers of instruction as provided by \cref{art:RegulationsConcerningFacultyStatus} \cref{sec:AppointmentReappointmentPromotionAndTenure}.\labelcref{sub:PartTimeVisitingAndOtherNonTenureTrackFaculty}.\labelcref{par:ReviewOfPartTimeVisitingAndOtherNonTenureTrackOfficersOfInstruction} of this Manual.

\end{enumerate}
\end{enumerate}

%___% 4. Committee on Committees
\paragraph{Committee on Committees}

\begin{enumerate}[label=\itemLevelA,ref=\itemRefA]
%____% a. 
\item Membership, Length of Term

The Committee on Committees shall be composed of the Secretary of the Faculty as Chairperson plus two faculty members elected for a two-year term.
%____% b. 
\item Responsibilities
\begin{enumerate}[label=\itemLevelB,ref=\itemRefB]
%_____% 1) 
\item The Committee shall implement election procedures, appoint faculty members to posts on certain committees for which election is not required, and see that each committee that does not have a chairperson convenes at the beginning of each academic year.
%_____% 2) 
\item The Committee shall also be responsible for receiving and recommending any proposed changes in Committee structure.

\end{enumerate}
\end{enumerate}

%___% 5. Curriculum Committee
\paragraph{Curriculum Committee}

\begin{enumerate}[label=\itemLevelA,ref=\itemRefA]
%____% a. 
\item Membership

The Curriculum Committee shall consist of the Dean of the College as Chairperson, the chairpersons of the four divisions, the Registrar as secretary, and one student. A member of LIS chosen by the Dean and the Dean of Academic Advising shall serve as non-voting advisors.

%____% b. 
\item Responsibilities
\begin{enumerate}[label=\itemLevelB,ref=\itemRefB]
%_____% 1) 
\item The Curriculum Committee shall supervise the structure and contents of the curriculum and shall implement current curricular policy.
%_____% 2) 
\item In cooperation with the divisions, which shall submit their proposed departmental and divisional offerings, it shall review and provisionally approve all course offerings.
%_____% 3) 
\item As the final authority for the curriculum is the responsibility of the Faculty as a whole, each semester the Committee shall present to the Faculty the proposed offerings for the following semester, for discussion of matters of substance and for final approval.
%_____% 4) 
\item The Committee shall study and take recommendations to the Faculty regarding innovations and proposed changes in specific major or minor programs, interdisciplinary, and divisional courses of study.

\end{enumerate}
\end{enumerate}

%___% 6. Educational Policy Committee
\paragraph{Educational Policy Committee}

\begin{enumerate}[label=\itemLevelA,ref=\itemRefA]
%____% a. 
\item Membership

The Educational Policy Committee shall consist of the Dean of the College, four members of the Faculty (one from each division), one of whom shall be chairperson, a member of the Library staff appointed by the Dean, and one student.
%____% b. 
\item Responsibilities
\begin{enumerate}[label=\itemLevelB,ref=\itemRefB]
%_____% 1) 
\item The Committee shall concern itself primarily with long-range planning.
%_____% 2) 
\item It shall keep the Faculty informed of new developments in the theory and practice in higher education generally, so as to maintain and foster a high standard at Wells College.
%_____% 3) 
\item It shall study and make recommendations to the Faculty concerning matters of educational policy and practice, such as calendar revisions, consortium relationships, and examinations.
%_____% 4) 
\item It shall collaborate with every committee with which it has a common concern, so as to coordinate long- and short-range planning most effectively.
%_____% 5) 
\item The Committee shall annually elect one of its faculty members to serve as representative to the Committee on Academic Affairs of the Board of Trustees and one member as budget representative to the Administration.

\end{enumerate}
\end{enumerate}

%___% 7. Faculty Evaluation Committee
\paragraph{Faculty Evaluation Committee}\label{par:FacultyEvaluationCommittee}

\begin{enumerate}[label=\itemLevelA,ref=\itemRefA]
%____% a. 
\item Membership
\begin{enumerate}[label=\itemLevelB,ref=\itemRefB]
%_____% 1) 
\item The Faculty Evaluation Committee shall consist of the four Division Chairpersons and one additional tenured faculty member from each of the four divisions, appointed by the Dean of the College in consultation with the Advisory Committee.
%_____% 2) 
\item Appointments to the Committee shall be for one year and shall be renewable indefinitely.
%_____% 3) 
\item The most senior faculty member from among the Division Chairpersons shall chair the Committee.
%_____% 4) 
\item No faculty member may serve simultaneously on the Faculty Evaluation Committee and either the Advisory Committee or the Review Committee.
%_____% 5) 
\item No individual shall serve as a voting member of the Committee or any sub-committee in the same academic year that they have  been or are to be evaluated for reappointment, tenure, or promotion. If a Division Chairperson is being evaluated, the Dean, in consultation with the Advisory Committee, will appoint a tenured faculty member from an appropriate area to take that Chairperson's place.
\end{enumerate}
%____% b. 
\item Responsibilities
\begin{enumerate}[label=\itemLevelB,ref=\itemRefB]
%_____% 1) 
\item The Committee shall meet with the Advisory Committee and Dean at least once a year to review the Criteria for Reappointment, Tenure, and Promotion and the procedures for the evaluation of Faculty.
%_____% 2) 
\item The Committee shall be divided annually into two sub-committees for functional purposes. The Dean shall make assignments to sub-committees in consultation with the Advisory Committee.
%_____% 3) 
\item Each sub-committee shall include two Division Chairpersons plus the representatives from the other two divisions. Each sub-committee shall be chaired by whichever Division Chair on the sub-committee is the most senior faculty member.
%_____% 4) 
\item Faculty members being considered for reappointment, tenure, or promotion shall be reviewed by the sub-committee that includes the chair of their division.
\end{enumerate}
%____% c. 
\item Procedure \label{item:procedure08}
\begin{enumerate}[label=\itemLevelB,ref=\itemRefB]
%_____% 1) 
\item The Advisory Committee, in consultation with the Dean, shall identify Faculty members not on the sub-committee but with significant teaching responsibilities in the major program(s) or teaching area of the Faculty member under review.
%_____% 2) 
\item These Faculty members shall be invited to provide information through any combination of group interviews, individual interviews, and/or written reports, as the sub-committee deems appropriate, but they will not participate in the deliberations or vote.
%_____% 3) 
\item The sub-committee shall gather advice from Faculty through any combination of group interviews, individual interviews, and/or written reports as the sub-committee deems appropriate.
%_____% 4) 
\item If the Evaluation Committee deems it necessary to consult with administrators or staff members, they may do so with permission of the Advisory Committee and the Dean.
%_____% 5) 
\item During its consideration of each individual Faculty member being reviewed, the sub-committee shall be augmented by the addition of the individual Faculty member's primary major program chair, if the chair is tenured. If the chair is not tenured or is the Faculty member being evaluated, a tenured Faculty member from the same or a related discipline shall be appointed by the Dean in consultation with the Advisory Committee. This additional member of the sub-committee shall have rights of discussion and vote in the sub-committee's review of that individual Faculty member. If the chair is already a member of the sub-committee, the Dean shall appoint an additional tenured Faculty member in consultation with the Advisory Committee.
%_____% 6) 
\item In all evaluations for tenure or promotion to full professor, and in any other case in which adequate disciplinary representation is not available within the College, an outside consultant from the candidate's academic discipline shall be appointed to assist the sub-committee in its deliberations. Consultants shall not vote.
%_____% 7) 
\item \label{item:results09} The results of the Evaluation Sub-committee's review shall be conveyed to the Dean of the College in writing, with copies to the Advisory Committee. The results of the Advisory Committee's review shall be conveyed to the Dean of the College in writing. After the completion of the evaluation, the Faculty member being evaluated shall receive copies of the Evaluation Sub- committee's letter and the Advisory Committee's letter. The Chair of the Evaluation Sub- committee shall receive a copy of the Advisory Committee's letter.

\end{enumerate}
\end{enumerate}

%___% 8. Inclusive and Intercultural Excellence and Off-Campus Committee
\paragraph{Inclusive and Intercultural Excellence and Off-Campus Committee}

\begin{enumerate}[label=\itemLevelA,ref=\itemRefA]
%____% a. 
\item Membership

The Inclusive and Intercultural Excellence and Off Campus Study Committee shall consist of three members of the Faculty, the Director of International Programs. A faculty member shall serve as chair.
%____% b. 
\item Responsibilities
\begin{enumerate}[label=\itemLevelB,ref=\itemRefB]
%_____% 1) 
\item The Committee shall serve as a vehicle for faculty action regarding inclusive and intercultural excellence at Wells College, incorporating global and domestic dimensions and their interrelations.
%_____% 2) 
\item The Committee's responsibilities shall include creating and implementing a campus-wide Strategic Plan on Inclusive and Intercultural Excellence as it relates to the academic program, providing planning assistance to other faculty committees, analyzing assessment data, and reviewing current and new off-campus study programs and general off-campus study polices.
%_____% 3) 
\item One faculty member of the committee shall serve as a representative to the President's Committee on Inclusive and Intercultural Excellence.

\end{enumerate}
\end{enumerate}

%___% 9. Review Committee
\paragraph{Review Committee}\label{par:ReviewCommittee}

\begin{enumerate}[label=\itemLevelA,ref=\itemRefA]
%____% a. 
\item Membership
\begin{enumerate}[label=\itemLevelB,ref=\itemRefB]
%_____% 1) 
\item The Review Committee shall consist of four tenured voting members of the Faculty who hold no administrative office, and who have been at Wells College at whatever rank for at least three years.
%_____% 2) 
\item At least one member shall be a woman and at least one a man.
%_____% 3) 
\item No more than one member of the same discipline shall serve at any one time (see \cref{sec:FacultyCommittees}\labelcref{sub:StandingCommittees}.\labelcref{par:AdvisoryCommitteeOnFacultyPersonnel}.\labelcref{item:membership06}.\labelcref{iitem:crossdiscipline07}).
%_____% 4) 
\item No faculty member may serve simultaneously on the Review Committee and either the Advisory Committee or the Faculty Evaluation Committee.
%_____% 5) 
\item A member of the Review Committee may not sit on any case in which they have been involved previously as member of the Advisory Committee on Faculty Personnel or the Evaluation Committee.

\end{enumerate}
%____% b. 
\item Responsibilities

A Faculty member may request that the Review Committee consider the recommendation made by the Advisory Committee and/or the Administration concerning their reappointment, promotion, or tenure. This request must be made in writing by the end of the semester following the recommendation. If the Review Committee determines that the Faculty member has shown good cause, it shall conduct a confidential review.
\begin{enumerate}[label=\itemLevelB,ref=\itemRefB]
%_____% 1) 
\item Review of Advisory Committee recommendations

\begin{enumerate}[label=\itemLevelC,ref=\itemRefC]
%______% a) 
\item The Review Committee shall meet in joint session with the Advisory Committee when deemed necessary by either committee.
%______% b) 
\item In the event the Review Committee concurs with the original recommendation of the Advisory Committee, both the Advisory Committee and the Faculty member granted a review shall be so informed.
%______% c) 
\item In the event the Review Committee's recommendation differs from that of the Advisory Committee, both committees shall, in joint session, discuss the case, vote collectively, and shall submit the result of such vote to the Dean of the College and the Faculty member requesting review.

\end{enumerate}

%_____% 2) 
\item Review of an Administrative recommendation
\begin{enumerate}[label=\itemLevelC,ref=\itemRefC]
%______% a) 
\item The Review Committee shall, if it so requests, be granted a hearing by the Board of Trustees.
%______% b) 
\item When there is disagreement between the Dean of the College or the President and the two committees acting jointly on any matter that would otherwise require the Advisory Committee's approval, the two committees shall, if they so request, be granted a hearing by the Board of Trustees. This provision shall not be taken as limiting the right of any member of either committee to submit to the President a minority report on any recommendation of the committees acting jointly and to have that report submitted to the Board of Trustees.

\end{enumerate}

\end{enumerate}
\end{enumerate}

%___% 10. Student-Faculty-Administration Board
\paragraph{Student-Faculty-Administration Board}

The duties, membership, and procedures of this Board are described in the \constitution. Its Faculty/Administration members are the Dean of the College, the Dean of Students, and the three faculty members of the Academic Standing and Advising Committee.

%__% C. Non-elected Standing Committees
\subsubsection{Non-elected Standing Committees}

%___% 1. Arts and Lectures Committee
\paragraph{Arts and Lectures Committee}

\begin{enumerate}[label=\itemLevelA,ref=\itemRefA]
%____% a. 
\item Membership

The Committee is composed of faculty, students and staff members interested in the selection, organization, and production of the College's Annual Arts and Lectures Series.
%____% b. 
\item Responsibilities

The Committee shall devise and propose a series of performing events and academic lectures for the college community. The Series annually brings to campus musicians, dancers, and actors that are recognized nationally and internationally. Traditionally the Series has consisted of one lecture, one dance performance, one musical performance and one theatrical performance in a given academic year. In addition, the committee sponsors off campus events such as ``Day on Broadway''.

\end{enumerate}

%___% 2. Community Court
\paragraph{Community Court}

Further information about the nature and functions of the Community Court can be found in the \constitution.
\begin{enumerate}[label=\itemLevelA,ref=\itemRefA]
%____% a. 
\item Membership, Length of Term

The Court is composed of ten students, three faculty members, and two staff members. One faculty member is selected each spring by the Committee on Committees for a three-year term.
%____% b. 
\item Responsibilities

The Community Court shall meet upon the request of a member of the student body, the Faculty or the Administration, to hear and decide cases of alleged violations of Community Honor as defined by the \constitution.
\end{enumerate}

%___% 3. The Judicial Appeals Board
\paragraph{The Judicial Appeals Board}

Further information about the nature and functions of the Judicial Appeals Board can be found in the \constitution.
\begin{enumerate}[label=\itemLevelA,ref=\itemRefA]
%____% a. 
\item Membership, Length of Term

The Judicial Appeals Board shall include the Dean of the College and the Dean of Students, both \exoff; four Faculty members, two as regular members of the Board and two as alternates; and four students, two regular and two alternates. The Committee on Committees shall select the Faculty members of the Board, one regular and one alternate each year. Terms run for two years, staggered.

%____% b. 
\item Responsibilities

The Judicial Appeals Board shall hear and decide appeals to decisions made by Community Court according to the procedures outlined in the Wells College Collegiate Bylaws and Constitution.

\end{enumerate}

%___% 4. Safety and Security
\paragraph{Safety and Security}
\begin{enumerate}[label=\itemLevelA,ref=\itemRefA]
%____% a. 
\item Membership

The Safety and Security Committee shall consist of two Faculty members, the Director of Security, who serves as chairperson, two students, and the Dean of Students.

%____% b. 
\item Responsibilities

The committee works in cooperation with the Director of Security to address student needs and concerns and to make recommendations to improve safety and security on campus.

\end{enumerate}

%_% Section 6. Faculty Role in Search Process for Senior Staff
\subsection{Faculty Role in Search Process for Senior Staff}
%__% A. 
\subsubsection{Participation}
Faculty shall serve on the search committee whenever a search is undertaken to fill a senior staff position (as of Spring 2012: President, Provost and Dean of the College, Chief Operating Officer, Vice President of External Relations, Dean of Students, and Director of Admissions).
%__% B. 
\subsubsection{Representation}
Faculty representatives shall enjoy full voice and vote and shall constitute at least half of the membership of search committees for Provost and Dean of the College, and the Dean of Students. Membership on a given search committee may be based on position, for example as representative from a Faculty Committee, however at least one member shall be elected from the faculty at large.

%%%%%%%%%%%%%%%%%%%%%%%%%%%%%%%%%%%%%%%%%%%%%%%%%%%%%%%%%%%%%%%%%%

%% ARTICLE II. REGULATIONS CONCERNING FACULTY STATUS
\section{REGULATIONS CONCERNING FACULTY STATUS}\label{art:RegulationsConcerningFacultyStatus}

%_% Section 1. Tenure, Professional Responsibility, and Academic Freedom
\subsection{Tenure, Professional Responsibility, and Academic Freedom}\label{sec:TenureProfessionalResponsibilityAndAcademicFreedom}

%__% A. 
\subsubsection{Tenure}

Wells College is committed to a model for faculty contracts that includes tenure as a basic component and that generally adheres to the guidelines of the \href{http://www.aaup.org/AAUP/}{American Association of University Professors} [AAUP]. When the policies and procedures in this manual differ from the AAUP guidelines, those of this \facman~ shall apply.

%__% B. 
\subsubsection{Professional Responsibility}
Wells College endorses the 1966 \href{http://www.aaup.org/AAUP/pubsres/policydocs/contents/statementonprofessionalethics.htm}{Statements on Professional Ethics} of the American Association of University Professors (AAUP Bulletin, Vol. 55, No. 1, Spring 1969, pp 86-87).

%__% C. 
\subsubsection{Academic Freedom} \label{sub:AcademicFreedom}
Wells College endorses the 1940 \href{http://www.aaup.org/AAUP/pubsres/policydocs/contents/1940statement.htm}{Statement on Academic Freedom and Tenure} of the American Association of University Professors. All members of the Faculty (See \cref{art:FacultyAndGovernance}, \cref{sec:Personnel}.\labelcref{sub:Faculty}) shall have full freedom of inquiry; of instruction subject to the educational plans of the Faculty and of their respective divisions; and of expression of opinion. A faculty member's exercise of the rights and duties as a citizen and member of a community shall in no way prejudice their academic status. In enjoying these rights a faculty member must recognize the corresponding obligation to accept full responsibility for their utterances and behavior.

%_% Section 2. Teaching Faculty
\subsection{Teaching Faculty}\label{sec:TeachingFaculty}
The teaching faculty of Wells College shall consist of two components:

%__% A. 
\subsubsection{Core}\label{sub:Core}
A core of faculty members consisting of those who are tenured and those in tenure-eligible positions.

%___% 1. 
\paragraph{Tenured} 
Tenured faculty are those who have had a successful tenure review and who have been awarded appointment without limit of time.

%___% 2. 
\paragraph{Tenure-Track} 
Faculty in tenure-eligible positions are those who have been appointed to tenure-track positions and who are reviewed according to the regulations in \cref{art:RegulationsConcerningFacultyStatus}, \cref{sec:AppointmentReappointmentPromotionAndTenure}. Review of tenure-eligible faculty shall result in renewal, non-renewal, or the granting of tenure, and/or promotion.

%___% 3. 
\paragraph{Size} 
The size and composition of the core faculty shall be based on the programmatic requirements necessary to attain and maintain the educational mission of the college as stated in the Wells College Mission Statement.

%__% B. 
\subsubsection{Contingent}
Faculty members appointed to full-time non-tenure-track visiting positions or part-time positions based on enrollment and programmatic needs.

%___% 1. 
\paragraph{Full-time} 
Faculty members in full-time non-tenure-track positions shall be given one or two-year contracts, either renewable or non-renewable and shall be limited to no more than six years.

%___% 2. 
\paragraph{Part-time} 
Part-time appointments shall be made for one semester, one year, or two years. Faculty members holding such positions shall not be eligible for tenure.

%_% Section 3. Appointment, Reappointment, Promotion, and Tenure
\subsection{Appointment, Reappointment, Promotion, and Tenure}\label{sec:AppointmentReappointmentPromotionAndTenure}
Faculty appointments, reappointments, and appointments with tenure shall be made in accordance with the provisions of this \namecref{art:RegulationsConcerningFacultyStatus}. The expectation should be that only the highest quality tenure-track faculty shall be reappointed and ultimately awarded tenure.

For an initial appointment for full-time visiting and tenure-track positions, the desirable characteristics of a candidate are excellence in teaching and scholarship, and the personal qualities that promise to make for a responsible and effective colleague.

For reappointment, promotion, and tenure, the emphasis of the evaluation shall be on teaching excellence, scholarly competence and activity, and contributions to the college community (e.g. in committees, in academic advising, in major field and divisional affairs, and in co-curricular activities).

%__% A. 
\subsubsection{Tenured and Tenure-Track Faculty}\label{sub:TenuredAndTenureTrackFaculty}

%___% 1. 
\paragraph{Initial Tenure-Track Appointments}
\begin{enumerate}[label=\itemLevelA,ref=\itemRefA]

%____% a. 
\item The Dean of the College, at the request of the division and in consultation with the Advisory Committee, shall form a Search Committee. This committee shall include divisional representation and members from the affected major field(s). It shall be chaired by the chairperson of the major field or the division.

%____% b. 
\item Interviews of candidates shall be conducted by the Search Committee, the Dean of the College, and whenever practicable the President.

%____% c. 
\item The Search Committee's recommendation shall be delivered by the chairperson to the Dean of the College.

%____% d. 
\item If the Dean agrees with the Search Committee's recommendation, the Dean shall forward their recommendation and the recommendation of the Search Committee to the President.

In cases where the Dean disagrees with the Search Committee's recommendation, the Dean shall meet with the Search Committee prior to sending the recommendations to the President. The President's decision shall then be made in consultation with the Dean and the Advisory Committee.

%____% e. 
\item Before offering the contract the Dean of the College shall consult with the Advisory Committee on the years of prior service to be counted toward the years of probationary service and also placement on the salary scale. Beginning with appointment to a tenure-track position, the probationary period shall not exceed seven years, including within this period full-time teaching in all institutions of higher education; except that, after a term of full-time service of more than three years at another institution, it may be agreed in writing that appointment is for a probationary period of not more than four years.

%____% f. 
\item The Dean shall provide the candidate with a written statement of the current status of the tenure policies at the College and the years of experience to be counted toward the years of probationary service.

%____% g. 
\item Initial tenure-track appointments shall ordinarily be for two years.
\end{enumerate}

%___% 2. 
\paragraph{Reappointment, Promotion, and Tenure}
\begin{enumerate}[label=\itemLevelA,ref=\itemRefA]

%____% a. 
\item Initiative for recommendations for reappointment, promotion, or tenure rests with the Dean of the College and the Advisory Committee. Recommendations regarding promotion may be initiated at the request of the Dean of the College, of the Advisory Committee, or of the faculty member concerned. Upon such initiative, the Faculty Evaluation Committee (\cref{art:FacultyAndGovernance}, \cref{sec:FacultyCommittees}.\labelcref{sub:StandingCommittees}.\labelcref{par:FacultyEvaluationCommittee}) shall conduct an evaluation.

%____% b. 
\item Faculty members being considered for reappointment, promotion, or tenure shall be reviewed by the subcommittee of the Evaluation Committee according to procedures in \cref{art:FacultyAndGovernance}, \cref{sec:FacultyCommittees}.\labelcref{sub:StandingCommittees}.\labelcref{par:FacultyEvaluationCommittee}.\labelcref{item:procedure08}.

%____% c. 
\item The Dean of the College's recommendation shall be delivered to the President with a copy to the Advisory Committee. In the event that the Dean's recommendation differs from the Advisory Committee's, the Dean shall consult with the Advisory Committee before delivering a recommendation to the President.

%____% d. 
\item Following receipt of the recommendations of the Dean and the Advisory Committee the President shall confer with the Dean and the Advisory Committee.

%____% e. 
\item The President's decision shall be made in consultation with the Dean of the College and they may issue a joint recommendation. The candidate shall receive copies of the Evaluation Committee, Advisory Committee and President/Dean's recommendation.
\end{enumerate}

%___% 3. 
\paragraph{Notice of Reappointment, Non-reappointment, Promotion and Tenure for Tenure-track Faculty Members}

\begin{enumerate}[label=\itemLevelA,ref=\itemRefA]

%____% a. 
\item Final decision with regard to any recommendation involving reappointment, non-reappointment, promotion, or tenure shall be made and notice of that decision sent in writing by December 15 for full- time officers of instruction in their second year of service, and no later than June 30 of the year prior to expiration of contract thereafter.

%____% b. 
\item Reasons for non-reappointment shall be conveyed in writing upon request of the faculty member concerned.

%____% c. 
\item A Faculty member may request that the Review Committee (\cref{art:FacultyAndGovernance}, \cref{sec:FacultyCommittees}.\labelcref{sub:StandingCommittees}.\labelcref{par:ReviewCommittee}) consider the recommendation made by the Advisory Committee and/or the Administration concerning their reappointment, promotion, or tenure. This request must be made in writing by the end of the semester following the recommendation. If the Review Committee determines that the Faculty member has shown good cause, it shall conduct a confidential review.

%____% d. 
\item Since the re-appointment of a faculty member shall not be refused on grounds which violate academic freedom (\cref{art:RegulationsConcerningFacultyStatus}, \cref{sec:TenureProfessionalResponsibilityAndAcademicFreedom}.\labelcref{sub:AcademicFreedom}), a member who alleges that their re-appointment has been refused on such grounds shall have the right to petition the Faculty for a hearing. Such a petition shall be submitted in writing to the Secretary of the Faculty and shall be read at the ensuing meeting of the Faculty. If the Faculty decides that the petition demonstrates sufficient grounds for such a hearing, it shall elect a five-member committee according to the provision of \cref{art:FacultyAndGovernance}, \cref{sec:CommitteeService}.\labelcref{sub:Elections}. The committee shall conduct a hearing according to the provisions of \cref{art:RegulationsConcerningFacultyStatus}, \cref{sec:Dismissal}. The Committee shall submit its opinion and the record of the hearing to the President and to the Trustees.
\end{enumerate}

%___% 4. 
\paragraph{Review of Tenured Faculty}\label{par:ReviewOfTenuredFaculty}
The Advisory Committee shall review the performance of each tenured faculty member at least once every seven years after the granting of tenure. The results of this review shall be conveyed in writing to the faculty member and the Dean of the College. The criteria to be applied on such reviews are the same as are applied to faculty being considered for appointment, reappointments, promotion, and tenure (see introduction to \cref{art:RegulationsConcerningFacultyStatus}, \cref{sec:AppointmentReappointmentPromotionAndTenure}). In the event the faculty member's performance is not satisfactory and there is a failure to improve over a four-year period, in which time there are at least two additional evaluations by the Advisory Committee, disciplinary proceedings for academic cause (as provided in \cref{art:RegulationsConcerningFacultyStatus}, \cref{sec:Dismissal}) or other appropriate measures shall be initiated by the Advisory Committee.

%__% B. 
\subsubsection{Part Time, Visiting and other Non-tenure-track Faculty}\label{sub:PartTimeVisitingAndOtherNonTenureTrackFaculty}

%___% 1. 
\paragraph{Procedures for Appointment}
The procedures listed above for appointments and recommendations need not apply when the appointment is for teaching fewer than four courses a year or for individuals hired as leave replacements. In such cases the Dean shall consult with the appropriate divisional chairpersons and major fields prior to making recommendations to the President. The Advisory Committee shall be informed of such recommendations and appointments.

%___% 2. 
\paragraph{Review of Part Time, Visiting and other Non-tenure-track Officers of Instruction}\label{par:ReviewOfPartTimeVisitingAndOtherNonTenureTrackOfficersOfInstruction}
The Advisory Committee shall review the performance of part-time officers of instruction who have served three years at Wells College. The review shall take place at least once every four years thereafter. All continuing, full-time, non-tenure-track faculty shall be reviewed at least every two years. The criteria to be applied in such reviews are the same as those applied in other faculty reviews, but with an emphasis on the three categories (teaching, scholarship, and community service) proportional to the nature of the position, and the procedures shall parallel those used in the seven-year evaluation of tenured faculty members. The Advisory Committee's recommendation shall be made to the Dean of the College, with copies to the divisional chairperson(s) and the faculty member concerned.

%___% 3. 
\paragraph{Expiration of Contract for Part Time and Visiting Faculty}
Contracts for faculty members in visiting appointments are assumed to expire at their end unless renewed in writing and, thus, no notice of non-reappointment is required.

%__% C. 
\subsubsection{Vacated Tenured or Tenure-Track Positions}

\paragraph{Process}

Whenever a tenured or tenure-track position has been vacated and the major field(s) determines that the position should continue:

\begin{enumerate}[label=\itemLevelA,ref=\itemRefA]
%___% 1.
\item  the chair of the major shall seek recommendation of the appropriate division(s), and then submit a request for replacement, along with the divisional recommendation(s), to the Educational Policy Committee and to the Dean of the College;

%___% 2.
\item  the Educational Policy Committee, taking into account the needs of the major(s) making the request, the recommendation of the division(s) involved, and the needs of the College academic program as a whole, shall recommend:
\begin{enumerate}[label=\itemLevelB,ref=\itemRefB]

%____% a. 
\item whether the position shall continue, be discontinued, or be reallocated to another field, and

%____% b. 
\item whether the position shall be tenure-track or non-tenure-track.
\end{enumerate}
The Educational Policy Committee shall forward its recommendation to the Advisory Committee, with a copy to the Dean.

%___% 3. 
\item The Advisory Committee shall review the personnel issues involved and if it has any observations forward them to the Dean, with a copy to the Educational Policy Committee.

%___% 4. 
\item The Dean shall forward a recommendation, along with the original request, the Educational Policy Committee's recommendation and the Advisory Committee's observations, to the President for action.
\end{enumerate}

%___% 5. 
\paragraph{Timeline} 
All actions on the request shall be made in as timely a manner as possible. Tenure-track positions shall be authorized as early as practicable in order to facilitate searches in national job markets.

%___% 6. 
The decision to designate a position as tenure-track shall have been made before hiring, not at the time of an interim reappointment or at the time of the tenure decision. However, the identification of a position as tenure-track does not imply that the individual appointed to that position shall qualify for tenure, but rather that the individual is eligible for consideration for tenure.

%_% Section 4. Academic Ranks
\subsection{Academic Ranks}\label{sec:AcademicRanks}

%__% A. 
\subsubsection{Appointments}
Faculty shall be appointed to one of the following ranks.

%___% 1. 
\paragraph{Lecturer}
Appointment as Lecturer normally shall be for up to one year and shall be reserved for those part-time faculty who hold no regular academic rank at Wells College.

%___% 2. 
\paragraph{Assistant Professor}
An initial appointment at, or promotion to, the rank of Assistant Professor normally shall be made for two years. An Assistant Professor shall ordinarily serve in that rank no more than six years. Qualifications include a doctorate or appropriate terminal degree in the field.

%___% 3. 
\paragraph{Associate Professor}
An initial appointment at the rank of Associate Professor normally shall be for two years, and when reappointment takes place, it shall be for two or three additional years.

%___% 4. 
\paragraph{Professor}
Initial appointments at the rank of Professor normally shall be for three years; by the end of the second year, the decision shall have been made either to grant tenure or to not reappoint.

%__% B. 
\subsubsection{Initial Appointment at the Rank of Professor or Associate Professor}
Normally no initial appointment at the rank of Professor or Associate Professor shall be made unless the recommendations of the Search Committee to the Advisory Committee and of the Advisory Committee to the Dean of the College indicate that the candidate appears to be of such promise as would, if fulfilled, justify the eventual granting of tenure.

%__% C. 
\subsubsection{Part-time}
Part-time officers of instruction may be appointed at the ranks of Assistant Professor, Associate Professor, or Professor under the provisions of this \namecref{art:RegulationsConcerningFacultyStatus} if their appointments are at least half-time in each academic year. For purposes of promotion, and eligibility for leaves, length of service shall be measured in units of half-time teaching equivalency. A change of a part time position to a full-time position shall require divisional, Advisory Committee, and Administrative approval as outlined in \cref{art:RegulationsConcerningFacultyStatus}, \cref{sec:AppointmentReappointmentPromotionAndTenure}.\labelcref{sub:TenuredAndTenureTrackFaculty}.

%__% D. 
\subsubsection{Shared Positions}
Shared Positions have been removed from the manual. Shared positions that were in effect at the time of this deletion remain in effect.

%_% Section 5. Leaves of Absence
\subsection{Leaves of Absence}\label{sec:LeavesOfAbsence}

%__% A. 
\subsubsection{Sabbatical Leave}

\paragraph{Eligibility and Conditions}
\begin{enumerate}[label=\itemLevelA,ref=\itemRefA]

%___% 1. 
\item Any tenured officer of instruction may apply for sabbatical leave, provided they will have completed six years of full-time service to the College prior to the year for which leave is being requested (except as provided for below.)

%___% 2. 
\item Faculty members tenured in their seventh year, taking their first sabbatical leave in their eighth year, shall be eligible for a second sabbatical leave in their fourteenth year.

%___% 3. 
\item Applications for Sabbatical Leave may be made with less than six years of full-time teaching equivalency with proportionally reduced compensation or released time. Such officers of instruction will also be eligible for certain specified fringe benefits, pro-rated where appropriate.

%___% 4. 
\item The granting of sabbatical leave at any time is contingent upon the promise of professional enrichment of the individual and the College, and upon the department's, Division's, and the College's interests not being impaired by the absence of the officer of instruction.

%___% 5. 
\item Sabbatical leaves shall be for either one-half year (that is, one semester plus the intersession) at full salary or for one full academic year at half salary.

%___% 6. 
\item If sabbatical leave is deferred at the express wish of the officer of instruction, another six-year waiting period begins immediately after the sabbatical leave is taken. If, however, the sabbatical leave is deferred for the convenience of the department, division, or College, this restriction shall not apply, and the subsequent sabbatical leave may be taken at the time when it would normally have been due had the deferral not occurred, provided that agreement in writing has been obtained at the time of postponement.
\end{enumerate}

\paragraph{Procedure}
\begin{enumerate}[label=\itemLevelA,ref=\itemRefA]
%___% 7. 
\item Application for leave shall be made no later than November 15, of the academic year preceding the one in which leave is to be taken. Such application, consisting of a proposed plan of study, as well as the department's and the division's recommendation (which should include a statement about the disposition of the applicant's courses and necessity of a replacement being engaged) shall be sent in writing to the Dean of the College, with copies of both letters being sent to the Advisory Committee. Whenever the chairperson of the division is the applicant, the senior member of the division apart from the chairperson shall normally be the division's agent in communicating the recommendation.

%___% 8. 
\item The Advisory Committee, the Dean of the College, and the President shall act on each application for leave in accordance with the basic pattern established in \cref{art:RegulationsConcerningFacultyStatus}, \cref{sec:AppointmentReappointmentPromotionAndTenure} for appointments, re- appointments, promotion, and tenure.

%___% 9. 
\item Decisions respecting leaves shall be communicated to the applicants no later than February 20 of the academic year preceding the one for which leave is requested; staffing decisions resulting from the granting of leaves shall be determined and the appropriate divisional chairs notified by the same date.

%___% 10. 
\item Three months before the beginning of the semester(s) in which leave is to be taken, the applicant shall file with the Advisory Committee the ``Agreement on Terms of Leave'' form, showing that agreement has been reached regarding the various fringe benefits. Copies of this form should also be given to the Treasurer and the Dean of the College, and one copy retained by the faculty member.

%___% 11. 
\item Upon returning to Wells, the sabbatical leave recipient shall file a report of the sabbatical activity with the Dean of the College.
\end{enumerate}

%__% B. 
\subsubsection{Special Leave}\label{sec:FacultyCommittees.B}
Special leaves may be granted to any officer of instruction or any other member of the Faculty at any time, as long as the interests of the department or the division or the College will not be impaired. No more than one year of special leave may be counted in the six years of full-time service needed to establish eligibility for, each sabbatical leave or in reckoning years of service for any other purpose. Special leave taken to complete requirements for a degree shall normally not be counted as service to the College.

%__% C. 
\subsubsection{Absences From Class}
Notification of frequent absence of members of the Faculty from appointed classes shall be made directly to the Dean of the College in the case of chairperson of divisions, and to the Dean through the chairpersons of divisions in the case of other members of the Faculty.

%_% Section 6. Retirement and Resignation
\subsection{Retirement and Resignation}

%__% A. 
\subsubsection{Notification}
An officer of instruction is expected to notify the Dean of the College of the intent to retire or resign by the end of the first (fall) semester of the prior academic year, if possible.

%__% B. 
\subsubsection{Emeriti}
A faculty member who has retired from Wells College in good standing may be awarded the title of Professor Emerita/Emeritus, provided that they have been employed full time by the College for at least ten years either as an associate professor or as a professor. The initiative for proposing emeritus status may originate with the President or the Dean of the College or the Advisory Committee. The designation requires the approval of the Dean of the College and the President and is awarded upon the vote of the Board of Trustees.

%_% Section 7. Dismissal
\subsection{Dismissal}\label{sec:Dismissal}
Wells College endorses the \href{http://www.aaup.org/AAUP/pubsres/policydocs/contents/statementon+proceduralstandardsinfaculty+dismissal+proceedings.htm}{STATEMENT OF PROCEDURAL STANDARDS IN FACULTY DISMISSAL PROCEEDINGS} of the American Association of University Professors. (1958)

%_% Section 8. Salary Policy
\subsection{Salary Policy}\label{sec:SalaryPolicy}

\subsubsection{Full Time Officers of Instruction}
%__% A. 
There shall be a salary scale proposed to the Board of Trustees each year by the President after consultation with the Advisory Committee.
%__% B. 
The actual salary scale in force in a given academic year shall be communicated annually to the Faculty at its September meeting, such scale to include the minimum, median, average, and maximum within each rank.

%__% C. 
\subsubsection{Part Time Officers of Instruction}
Salaries for part-time officers of instruction and for those holding titles other than Lecturer, Assistant Professor, Associate Professor, and Professor shall be determined by the Dean of the College in consultation with the Advisory Committee and shall be based on their previous experience and rank and on the fraction of a normal teaching load that the individual is teaching. Provisions for the Campbell Visiting Professor and the Dean of the College are expressly excluded from the requirements of this \namecref{sec:SalaryPolicy}.

%_% Section 9. Taking of Courses for Credit or Audit
\subsection{Taking of Courses for Credit or Audit}
Faculty members and their partners may audit or take for credit any course regularly offered by the College with prior permission of the instructor concerned and the Dean of the College. See also \cref{art:RegulationsConcerningStudentStatus}, \cref{sec:NonTraditionalStudents}.\labelcref{sub:SpecialStudents}.\labelcref{par:CategoriesOfSpecialStudents} \labelcref{item:employeeClasses} \& \labelcref{item:spouseClasses}.

%_% Section 10. Distribution of the Faculty Manual
\subsection{Distribution of the Faculty Manual}
Each Faculty member, each member of the Board of Trustees, and the President of Collegiate Association, shall be provided with a copy of the \facman. In addition, a copy shall be placed on reserve in the Library. At the beginning of each academic year, the Secretary of the Faculty shall prepare new pages embodying those amendments to the Manual adopted during the previous academic year, and provide for their distribution.

%_% Section 11. Revision of this Article
\subsection{Revision of this Article}\label{sec:Revision}
Amendment of this \namecref{art:RegulationsConcerningFacultyStatus} may be proposed, or discussion thereon initiated at any time by the Trustees, the President, the Dean of the College, or any other member of the Faculty. Any amendments, to be effective, must have the approval of both the Faculty and the Board of Trustees. See also \cref{art:Amendment}. This \namecref{art:RegulationsConcerningFacultyStatus}, or any amendment to it, shall not impair written contracts made between the College and a member of the Faculty prior to the adoption of the Article or any pertinent amendment.

%%%%%%%%%%%%%%%%%%%%%%%%%%%%%%%%%%%%%%%%%%%%%%%%%%%%%%%%%%%%%%%%%%%%%%%






















































%% ARTICLE III. REGULATIONS CONCERNING STUDENT STATUS
\section{REGULATIONS CONCERNING STUDENT STATUS}\label{art:RegulationsConcerningStudentStatus}

%ARTICLE IV. REQUIREMENTS FOR DEGREES Section 1. Bachelor of Arts
\subsection{Requirements for Degrees}

\subsubsection{Bachelor of Arts}

%A.
\paragraph{Course Requirements}

\begin{enumerate}[label=\itemLevelA,ref=\itemRefA]
%1
\item Each student must successfully complete a minimum of 120 semester\modified{5/13/93} hours of college level credit.
%2
\item At least 60 semester hours must be taken in residence at Wells College\modified{5/13/93} or in Wells College affiliated programs. While students may take more than 20 semester hours in Wells off-campus programs, no more than 20 semester hours will be counted toward the 60 semester hour requirement.
%3.
\item At least 6 courses (3-4 semester hours each) in a major must be taken at Wells College or through an affiliated program. At least 4 courses (3-4 semester hours each) for a minor must be taken at Wells College.
%4.
\item Each student, whether first-year or transfer student, is expected to satisfy College requirements, the requirements of the major, and the requirements \modified{4/8/97} of any minors as stated in the catalog in effect at the time of first matriculation at Wells College. A student who ceases to attend the College for two years or more will be expected to meet the requirements as stated in the catalog in effect at the time of return.
%5.
\item No more than 45 semester hours in any one discipline will be included \modified{5/13/93}  in the 120 semester hours required for the degree.
%6.
\item Each student will maintain a minimum Grade Point Average of 2.0 for all Wells and Wells-affiliated courses. See \cref{art:RegulationsConcerningStudentStatus}.\labelcref{sec:AcademicRegulations}.\labelcref{sub:AcademicStanding}.\labelcref{par:UnsatisfactoryProgress} for the consequences of grade point averages which fall below 2.0.
%7.
\item Students may earn no more than 4 semester hours of credit during any given January  Intersession.\modified{2/8/05}
\end{enumerate}

%B.
\paragraph{Senior Residency Requirement}

All seniors, except those in Dual Degree Programs, will be in residence at Wells. Those students in Dual Degree Programs will be in residence at Wells during their junior year.

\oldbreak{IV-I}

%C.
\paragraph{Major Field Requirements}

Each \modified{5/13/93}  student must meet the course requirements for a major, including a comprehensive evaluation in the major, and maintain at least a 2.0 GPA for all Wells or Wells-affiliated courses taken for the major field during the sophomore, junior and senior years. No major will require more than 65 semester hours.

%D.
\paragraph{Time Limit}

Each student is expected to complete the requirements for the degree, Bachelor of Arts, within seven years of matriculation at Wells College. 

%E.
\paragraph{Exceptional Cases}

In exceptional cases, upon recommendation of the student's academic advisor, the Committee on Academic Standing and Advising may waive one or more of these requirements.

%F.
\paragraph{Degree Expected}

 Except\modified{9/9/97}  for students in 3/2 or 3/4 articulation programs, a student with senior standing may petition the Committee on Academic Standing and Advising to participate in commencement activities if all requirements for the major, the senior comprehensive evaluation, and at least 114 semester hours will have been completed. The student would be expected to complete the remaining work by the end of the calendar year. A student in a 3/2 or 3/4 articulation program will be expected to follow the terms of the program and may participate in commencement activities when the work at Wells College is completed.

%Section 2. Master of Arts
\subsubsection{Master of Arts}

\paragraph{Admission} Admission to graduate study does not constitute admission to candidacy for a Master of Arts degree. Application for candidacy for the Master of Arts degree must be made upon the accumulation of twelve hours of graduate credit, transferred to or earned at Wells College.

\paragraph{Requirements for the Degree:}

\begin{enumerate}[label=\itemLevelA,ref=\itemRefA]

%A.
\item Candidates for advanced degrees must be graduates of an approved college or university and must satisfy the departments concerned of their fitness to complete a proposed program.

%B.
\item A minimum of two semesters in residence or its equivalent.

%C.
\item Eight semester courses (24 hours) of work of advanced character, a thesis and its oral defense; or ten semester courses (30 hours) of work of advanced character and general examinations, written and oral. Students taking general examinations at the end of a semester are excused from final examinations in the courses taken during the same semester.

\oldbreak{IV-2}

%D.
\item A maximum of ten hours may be transferred from accredited institutions.

%E.
\item A qualified Wells College junior or senior may be permitted to accumulate a maximum of twelve graduate credits.

%F.
\item Grades of B- or better in all courses offered for the degree.

%G.
\item A reading knowledge of at least one foreign language is required.

\item No more than seven years shall elapse between admission to candidacy and completion of the requirements for the degree.

\end{enumerate}

\oldbreak{IV-3}



%ARTICLE V. REGULATION OF STUDIES
\subsection{Regulation of Studies}\label{sec:RegulationOfStudies}

%Section 1. Choice of Studies
\subsubsection{Choice of Studies}\label{sub:ChoiceOfStudies}

%A.
\paragraph{Responsibility}

A student shall be held responsible for observing the requirements for the degree.

%B.
\paragraph{Declaration of Major or Minor}

\begin{enumerate}[label=\itemLevelA,ref=\itemRefA]
%1.
\item At any time after elevation to sophomore standing, but no later than\modified{11/12/01} March 1 in the sophomore year, a student shall declare a major.

\item To be eligible to declare the major, the student must have a GPA of 2.0\modified{5/13/93} in course work in the proposed major. If fewer than 6 semester hours in the proposed major have been completed, a student may petition for provisional approval contingent on a GPA of 2.0 in course work in the proposed major at the end of the sophomore year. A transfer student with junior standing shall declare a major by the end of the first semester; formal admission into the major requires the written approval of a Wells College faculty member in the major based on evaluation of the transfer record.

\item Students are not normally encouraged to elect double majors. If a student \modified{10/17/00} feels that their plans necessitate a double major rather than a major and a minor,  a Double Major Proposal form must be filed with the Registrar no later than the \modified{9/9/2003}
%last day of classes of the junior year (Proposed May 2003)  -BADAMS via DKOESTER
end of advising week of the first semester of the junior year. This Proposal must indicate how the student's proposed plan of study meets the College's goals for a sound liberal arts program, including courses that satisfy the Core Requirements. The Proposal must be approved by an advisor from each major. The student must have a cumulative grade point average of at least 3.5 at the time of the proposal. A double major may not include an individualized major. Final approval rests with the Academic Standing and Advising Committee.

%2.
\item A student desiring a minor shall declare the minor on a Declaration of Minor\modified{4/8/03} form, available in the Registrar's office or online, no later than\modified{5/08/12} the last day of classes of the first semester of the senior year. The declaration must be approved by the student's major advisor and the faculty coordinator for the minor.

\item A student may request to complete two minors by filing two minor declaration forms, available in the Registrar's Office or online, no later than the last day of classes of the first semester of the senior year. Each form must be approved by the student's major advisor and the faculty coordinator for the minor.
\end{enumerate}

%C.
\paragraph{Course Loads and Limits}

\begin{enumerate}[label=\itemLevelA,ref=\itemRefA]
\item  The normal course load is 30 semester hours per year. 
\item The minimum course load         \modified{4/12/98}  full-time student in a semester is 12 semester hours. 
\item There is a limit of 8 semester hours of credit toward the minimum 120 semester hours of credit required for graduation for courses that bear less than 3 semester hours of credit in the following two categories: (1) physical education courses, and (2) arts and performance courses, except those taken to meet requirements for the major or minor. 
\item There is a limit of 12 semester hours of credit for independent study and 12 semester hours of credit for internships that can count toward the minimum 120 semester hours of credit required for graduation. 
\item When a course in the regular curriculum is taken as an independent study (designated by the regular course number followed by ``I''), it does not count toward the 12-semester-hour limit on independent study.
\end{enumerate}

%D.
\paragraph{Acquiring Credit}\label{par:AcquiringCredit}

\begin{enumerate}[label=\itemLevelA,ref=\itemRefA]
\item \label{iitem:incomingCredit} Any incoming student will acquire credit toward the Wells degree by:

\begin{enumerate}[label=\itemLevelB,ref=\itemRefB]
%1. 
\item \label{iitem:passing} Passing \modified{9/14/99} a course in the Liberal Arts with a grade of C- or better, taken from an accredited college or university. Students entering Wells College under an articulation agreement will be granted credit according to the agreement. (See \cref{art:RegulationsConcerningStudentStatus}, \cref{sec:AdmissionOfStudentsToAdvancedStanding}.\labelcref{sub:AssignmentOfCredit})
%2.
\item \label{iitem:articulation} Entering with an A. A. degree under an articulation agreement. Such students will be held to the terms of that agreement.
%3
\item \label{iitem:ap} Scoring a grade of four (4) or five (5) on the Advanced Placement exam in any field in which Wells courses are offered. The appropriate discipline will consider a grade of three (3) on an individual basis at the request of the student. Semester hours of credit will be awarded in accord with Advanced Placement recommendations. For similar external\modified{11/14/95} programs, credit will be awarded at the discretion of ASA, with appropriate consultation with relevant disciplines; applicability to a major/minor or placement in advanced courses rests with the major advisor/minor coordinator. Each program is to be reviewed individually.
%4.
\item \label{iitem:clep} Passing a CLEP student examination in a subject covered in the Wells College curriculum with at least the minimum score recommended by the American Council of Education. Normally no more than two courses (6-8 semester hours) of such credit may be applied toward a Wells degree.
%5.
\item \label{iitem:placement} At the discretion \modified{5/8/90}of the instructor, taking and passing an examination designed by the instructor of a Wells course to cover the material of that course. Normally no more than two courses (6-8 semester hours) of such credit may be applied toward a Wells degree. Examination so requested must be taken during the student's first year of attendance and must be taken before further credit is completed in the discipline.
%6. 
\item \label{iitem:professional} Submitting material \modified{5/13/93} in evidence of previous professional experience (paid and non-paid) for approval for internship credit by the relevant discipline internship coordinator and by the Academic Standing and Advising Committee. Such approval is normally given only for work completed after high school gradation and is based on written material submitted by the student, describing in detail the experience gained and its relationship to their academic work, and a letter of evaluation from a supervisor or employer. Credit so earned is limited to a maximum of two internships (6-8 semester hours).
%7.
\item \label{iitem:portfolio} Submitting portfolios that demonstrate learning and document experience. Such portfolios shall be presented, developed, and articulated in consultation with a faculty member. Portfolios approved by the faculty member shall be submitted to the Academic Standing and Advising Committee for approval for credit. Credit so earned is limited to a maximum of two courses (6-8 semester hours).
\end{enumerate}

\item In all cases, placement in advanced coursed and applications to the requirements for the major will be at the discretion of the major field advisor.
\item Normally a maximum of nine courses (27-36 semester hours of credit) earned in categories \labelcref{iitem:passing,iitem:articulation,iitem:ap,iitem:clep,iitem:placement,iitem:professional,iitem:portfolio} will be accepted for a Wells degree.
\item No grades will be entered on a student's Wells College record for semester hours earned through means \labelcref{iitem:passing,iitem:articulation,iitem:ap,iitem:clep,iitem:placement,iitem:professional,iitem:portfolio}.
\end{enumerate}

%E.
\paragraph{Accelerated Program}
A student may accelerate the normal course program with the object of graduating in less than four (4) years by submitting a program for approval to the Dean of the College.

%F.
\paragraph{Individualized Major}

\begin{enumerate}[label=\itemLevelA,ref=\itemRefA]
\item  Students may propose a self-designed Individualized Major if they have an educational objective \modified{11/10/2009} that would be better served than by pursuing one of the established majors at Wells College or by pursuing an established major and minor.  
\item Each Individualized Major is expected to meet the philosophical and educational goals of Wells College and to afford the student maximum exposure to the breadth and depth of a liberal education as well as an opportunity to undertake advanced work in an area of special interest.  
\item The Individualized Major must have a clear focus, and at least two disciplines must be substantially represented. 
\item The Academic Standing and Advising Committee will approve or reject these programs.
\end{enumerate}

%G.
\paragraph{Language}
 A student is expected to use clear and idiomatic English in all classroom work and examinations. If found deficient in this respect,  extra work in English composition under the direction of the Department of English may be required. This extra work shall not count towards the degree.

%H.
\paragraph{Auditing Courses}
\begin{enumerate}[label=\itemLevelA,ref=\itemRefA]
\item A student may visit a course on mutual agreement with the instructor. 
\item A student who registers for an ``audit'' must participate actively in the course and must complete all work specified  by the instructor at the beginning of the course.
\item  A student desiring to audit a course shall get permission of both the faculty advisor and the instructor of the course.
\item Under either of the following circumstances, a student must petition the Dean of the College:  to audit more than one course in a semester or if the request to audit a course is made after the first week of classes. 
\item Normally, petitions to change the status of a course to an audit will not be approved after the sixth week of classes, unless there are medical reasons.
\end{enumerate}

%I.
\paragraph{Off-Campus Study}
 Any student whose academic progress will be advanced by a semester or a year spent at another college or university, and whose record testifies to the ability to profit from such study, may request permission to undertake it by applying to the Inclusive and Intercultural Excellence and Off-Campus Study Committee.

%J.
\paragraph{Summer credits}
Normally students may take no more than 8 semester hours of courses (such as: courses taken at another institution or independent study courses with Wells faculty) in any one summer. Students are expected to apply for prior approval for summer courses taken elsewhere. They are subject to the limitation on registration imposed by the institution they are visiting.\modified{5/13/93}

\oldbreak{V-3}

%Section 2. Registration for Courses.
\subsubsection{Registration for Courses}

%A.
\paragraph{Selection}
Schedules, showing courses selected for the oncoming semester, shall be submitted at times designated by the Registrar. The student may submit the schedule only after being granted permission by their Academic Advisor.

%B.
\paragraph{Add/Drop}
During the first week of classes of either semester, students will be permitted to change their schedule of courses without petition.

%C.
\paragraph{Late Registration}
Each student is responsible for submitting a semester program and schedule on or before the deadline fixed by the Registrar. A late registration service fee will be incurred by any student who does not meet the stated deadline unless  exempted by the Dean of the College upon written request of the student, stating the reasons. A receipt for the fee or an email from the Treasurer's Office must be submitted before the hold is released. The fee does not apply to revisions of previously submitted semester programs and schedules that are made during the time periods established for such revisions.

%Section 3. Courses for Grade of \textbf{Pass }or Fail.
\subsubsection{Courses for Grade of \textbf{Pass} or Fail}\label{sub:CoursesForGradeOfPassOrFail}

\paragraph{Designation}
At the time of registration, or within the first week of classes of either semester, students may designate courses to be graded Pass-Fail. A freshman may elect one course on a Pass-Fail basis, but only in the Spring Semester . A sophomore, junior, or senior may choose two courses each year for which  a grade of Pass or Fail will be received. 

\paragraph{Regulations}
The following regulations apply:

\begin{enumerate}[label=\itemLevelA,ref=\itemRefA]
%1.
\item After the first week of classes in either semester, students must petition the Dean of the College to change the basis of grading a course, either from a letter grade to a grade of pass or fail, or from a grade of pass or fail to a letter grade. After the sixth week of classes, such petitions will be approved only in unusual circumstances, such as when there are medical reasons.

\item Courses so chosen shall not be taken at the same time.

%3. 
\item Courses so chosen shall not be in a student's major field.

%4.
\item Courses may be graded Pass with Distinction, Satisfactory, Fail, but must be so designated in the catalog description of the course. If a student takes more than two courses so graded, the additional course or courses shall be considered as courses taken under the Pass-Fail option (see above), except that such courses may be used to satisfy the requirements of a student's special field.

%5.
\item In the computation of grade point averages, all courses taken on a Pass-Fail or similar basis will be disregarded by the Registrar except for the grades of PD and F, which will have the numerical equivalents of 4.0 and 0, respectively.

%6.
\item One course so chosen may be used to satisfy the requirements of the minor field, but only at the 100-level.

\end{enumerate}

\oldbreak{V-4}

%('24/89]Section 4. Calculation of Averages
\subsubsection{Calculation of Averages}\modified{/24/89}

The calculation of a student's grade point average will be based on all the Wells courses and Wells-affiliated courses undertaken following  initial matriculation at Wells College, with these exceptions:

\begin{enumerate}[label=\arabic*]
%1.
\item Courses in which a grade of P or S has been earned will be handled as noted in \cref{sub:CoursesForGradeOfPassOrFail} above.

%2.
\item Students may repeat only courses for which they have previously earned a grade\modified{4/8/03}  of ``F'' (or ``U''), or courses that are designated ``repeatable.'' In the case of a course not designated ``repeatable,'' and for which the student received a grade of ``F'' or ``U,'' after the repeat both grades will be posted on the transcript and be reflected in the respective semester (term) grade point averages. However, only the higher grade earned will be calculated into the cumulative grade point average. In the case of a course designated ``repeatable,'' every grade earned will be posted on the transcript and be calculated into the semester (term) and cumulative grade point averages. This is true even when a grade of ``F'' or ``U'' is received.

%3.
\item Courses taken at other institutions, except in Wells-affiliated programs, will not be included in the calculation of the grade point average as a matter of normal practice.

\end{enumerate}

%Section 5. Internships              [5/13/93]
\subsubsection{Internships}\modified{5/13/93}

Any paid internship must be approved by the faculty sponsor, faculty coordinator of internships, and the Director of Experiential Learning and Career Services.

%Section 6. Withdrawal from Courses 
\subsubsection{Withdrawal from Courses}

%A.

\paragraph{Full Semester Courses}

\begin{enumerate}[label=\itemLevelA,ref=\itemRefA]
%1. 
\item The drop/add period for full semester courses shall be the first ten class days of the semester. When a course is dropped, no record of the course will appear on the transcript.\modified{2/08/11}

\item After the end of the drop/add period and before the end of the ninth \modified{2/08/11} week of the semester, a student may withdraw from any course by consulting with the academic advisor and the course instructor, and by filing the appropriate form with the Registrar. In such cases the Registrar shall record the withdrawal on the student's transcript with a W, but shall record no grade.

\oldbreak{V-5}

%3.
\item After the ninth week of the semester, a student who wishes to withdraw from a course\modified{2/03/11} must petition the Dean of the College. If the petition is approved, the Registrar shall record the withdrawal with the indication of W if the student was passing the course at the time of withdrawal, or WF if the student was failing the course at the time of withdrawal, except in cases in which --- by the judgment of the Dean of the College --- the withdrawal was in effect required for medical or other grave personal reasons. In such cases the Registrar shall record W.

%C
\item No petition for withdrawal will be accepted after the last day of classes. 

\end{enumerate}

%B.
\paragraph{Half Semester Courses} \modified{5/13/93}

\begin{enumerate}[label=\itemLevelA,ref=\itemRefA]
%1.
\item The drop/add period for half semester courses shall be the first week the class meets.

%2.
\item After the end of the first week and before the end of the third week the class meets, a student may withdraw from the course with the provisions as given above.

%3.
\item  After the third week the class meets, a student must petition the Dean of the College to withdraw from the course, with the provisions as given above.

%C
\item No petition for withdrawal will be accepted after the last day of classes. 
\end{enumerate}

\oldbreak{V-6}

%ARTICLE VI. INDEPENDENT STUDY AND TUTORIALS
\subsection{Independent Study and Tutorials}\label{sec:IndependentStudyAndTutorials}

%A.
\subsubsection{Independent Study}

\paragraph{Purpose}

\begin{enumerate}[label=\itemLevelA,ref=\itemRefA]
\item Independent Study is an option for those students who have demonstrated an ability to\modified{5/13/93} work without close supervision. 

\item The purpose of Independent Study is to supplement the more structured methods of regular courses with the opportunity for the student of high initiative and responsibility to apply their abilities to new material with a minimum of guidance. 

\item While there are no formal course prerequisites for such work, it is the normal expectation that independent projects will involve explorations in depth of some prior work, or work for which the student has special non-academic background that qualifies them\modified{10/10/89} for the study proposed. Exceptions may be made in cases where a student of generally demonstrated capability wishes to do exploratory work in a field in which they have little or no background, in a manner not provided for by the regular curriculum. 

\item Independent\modified{5/12/98} study normally is conducted at the 300 level as XXX 399. Under exceptional circumstances and at the discretion of the instructor, preferably in the summer or the January Intersession, students may undertake independent study with the designation XXX 199 or XXX 299, to indicate that the work is not at a level sufficiently advanced to warrant a 300-level designation.

\item In cases of irreconcilable schedule conflicts, a course from the regular curriculum can be taken as an independent study, either with the designation, 199/299/399, or with the regular course number, at the discretion of the instructor.
\end{enumerate}

\paragraph{Approval}
\begin{enumerate}[label=\itemLevelA,ref=\itemRefA]
\item A qualified Sophomore, Junior or Senior may pursue one Independent Study Project for one to three semester hours during any semester. An exceptionally able and responsible student may pursue such work outside the academic semester. Under exceptional circumstances first year students may be permitted by petition to the Committee on Academic Standing and Advising to pursue an Independent Study Project after their first semester, either over January or in the Spring Semester.

\item A student who has a plan for Independent Study should consult with the instructor under whose guidance the work is proposed. If the instructor approves the project, a brief description of the project and the amount of credit proposed for it, approved by the instructor and advisor, must be submitted to the Registrar with the program of study during the registration period preceding the semester in which the work is to be conducted. 

\item Final approval of all projects rests with the Academic Standing and Advising Committee; copies of the proposal will be distributed to the instructor, advisor, and student. The student should not assume approval until formally so notified.
\end{enumerate}

%B.
\subsubsection{Tutorials}

\paragraph{Purpose}
\begin{enumerate}[label=\itemLevelA,ref=\itemRefA]
\item A tutorial offers a means for in depth study of material or a topic not otherwise covered in the curriculum.  

\item Tutorials are normally for one credit hour and normally involve a small group of students meeting together with a faculty member for one hour per week.  The tutorial thus offers a flexible mechanism for providing instruction on topics that may arise via student request, to meet special or last minute curricular needs, and/or for skill development (such as in the performing arts).                                                                                                                     
\end{enumerate}

\paragraph{Approval }
\begin{enumerate}[label=\itemLevelA,ref=\itemRefA]
\item In the case where a tutorial is offered for the first time, the instructor shall submit a Tutorial Proposal Form, which will include a description of the course, a synopsis of the topic(s) to be covered, examples of resources to be utilized, and the requirements and expectations to be placed on the students, including a clear method of evaluation.  This proposal should be signed by the instructor and by the Chair of the Major Field (indicating support by the major field), and submitted to the Registrar as soon as possible, but not later than the second week of classes in the semester the tutorial is being offered.  

\item Like Independent Studies, final approval of the tutorial rests with the Academic Standing and Advising Committee.

\item A tutorial being offered for the second time, or any tutorial that is to be required as part of a major or minor program, shall be subject to the full peer approval process as any other course.  Once approved by this means no further approval is required for future offerings.  A tutorial being offered for the first time may go through the full approval process, but is not required to do so.
\end{enumerate}

\paragraph{Grading and Credit}
\begin{enumerate}[label=\itemLevelA,ref=\itemRefA]
\item Tutorials may be graded on a letter grade basis or S/U.

\item Students may repeat a particular tutorial for credit if the focus of the course is significantly different or if it allows for the refinement and continued development of a skill, e.g., in the performing arts.  

\item No more than 8 semester hours of tutorial credit may count towards the 120 semester hours required for graduation.
\end{enumerate}

\oldbreak{VI-2}

%ARTICLE VII. ACADEMIC REGULATIONS
\subsection{Academic Regulations}\label{sec:AcademicRegulations}
%Section 1. Academic Standing 
\subsubsection{Academic Standing}\label{sub:AcademicStanding}

\paragraph{Class Designation }

%1.
\begin{enumerate}[label=\itemLevelA,ref=\itemRefA]
\item Determination

Class designation is determined by the number of courses a student has completed satisfactorily: Freshman, 0-29 semester hours; Sophomore, 30-59 semester hours; Junior, 60-89 semester hours; Senior, 90 -or more semester hours.

%2.
\item Course Deficiencies



A student wishing to remove a course deficiency by taking a course at another institution shall have that course approved in advance by the Dean of the College, in consultation with an appropriate Faculty member. Criteria for accepting a course will be the same as for assignment of credit for admission to advanced standing, \cref{art:RegulationsConcerningStudentStatus}, \cref{sec:AdmissionOfStudentsToAdvancedStanding}.\labelcref{sub:AssignmentOfCredit}.
\end{enumerate}
%B.
\paragraph{Academic Honors}

\begin{enumerate}[label=\itemLevelA,ref=\itemRefA]

%1.
\item Dean's List

At the end of each semester the Registrar shall prepare a Dean's List of those students who were registered for at least 12 semester hours of letter-graded courses and earned an average of 3.5 or higher for that semester. Courses taken under the exchange agreement with Cornell University shall be included in the GPA calculation for the Dean's List. A student with a grade or I or NR shall not be on the Dean's List. This honor will be recorded on a student's permanent record.

%2.
\item Graduation Honors  \modified{5/13/97}

\begin{enumerate}[label=\itemLevelB,ref=\itemRefB]

%a.
\item Latin Honors: traditional Latin Honors, \textit{cum laude, magna cum laude, }and \textit{summa cum laude, }are awarded at graduation for excellence in course work throughout a student's college career. The degree of Bachelor of Arts is awarded \textit{cum laude }to those who have a grade point average of 3.50-3.749; \textit{magna cum laude }to those who have a grade point average of 3.75-3.899; and \textit{summa cum laude }to those who have a grade point average of 3.90 and above. All Wells and Wells-affiliated courses are included in these calculations.



\oldbreak{VII-1}

\item Distinction: The degree will be awarded \textit{with distinction in the major field }to each student who is recommended for the award by the faculty in the major field because the student has shown:

\begin{enumerate}[label=\alph*)]

%(1)
\item Outstanding ability (3.50 average) in major field course work done in the sophomore, junior, and senior years;

%(2)
\item the capacity to do independent work with a high degree of initiative, genuine intellectual curiosity, and sense of responsibility-matters to be determined solely by the department which is recommending distinction;

%(3)
\item and distinction in the comprehensive evaluation.
\end{enumerate}

Acting on departmental recommendations, the Committee Academic Standing and Advising will transmit names of candidates for distinction to the Faculty.

\item Graduation honors will be noted on a student's permanent record.
\end{enumerate}

\end{enumerate}

%C.
\paragraph{Unsatisfactory Progress}\label{par:UnsatisfactoryProgress}

\begin{enumerate}[label=\itemLevelA,ref=\itemRefA]
%1.
\item Registrar's List \modified{4/14/09}

      After the close of each semester, the Registrar will prepare the Registrar's List for the use of the        Committee on Academic Standing and Advising. The Registrar's List will consist of the names of        students whose cumulative grade point averages are below 2.0.

%2.
\item Faculty Review \modified{4/14/09}        

The Committee on Academic Standing and Advising shall review student records after each semester to ascertain if students are achieving a cumulative grade point average of at least 2.0. A student whose cumulative grade point average is below 2.0 shall be so notified by the Dean of the College or by the Dean for Academic Advising.

%3. 
\item Determination of Academic Standing \modified{4/14/09}       

\begin{enumerate}[label=\itemLevelB,ref=\itemRefB]
\item The Committee on Academic Standing and Advising will normally issue an Academic Warning to any student whose cumulative grade point average falls below 2.0 for the first time. 
\item The Committee will normally issue an Academic Probation to any student whose cumulative grade point average falls below 2.0 for the second time, and a second Academic Probation to any student whose cumulative grade point average falls below 2.0 for the third time. 
\item For a student whose cumulative grade point average falls below 2.0 for a fourth time, the Committee on Academic Standing and Advising will normally issue an Academic Suspension for a minimum period of two semesters. In addition, the Committee on Academic Standing and Advising will consider any student for suspension at any time\modified{5/08/12} who is deemed to have failed to make satisfactory progress toward the degree, for example, a student whose cumulative grade point average is below 0.5 after two semesters of study at the College. Suspended students must apply for readmission to the Dean of the College. Readmission is at the discretion of the Dean of the College. The Dean of the College may set expectations that the student must satisfy upon return, based on the student's academic progress thus far.
\end{enumerate}

% 4. 
\item Permanent Dismissal for Academic Reasons \modified{4/14/09}      

The Dean of the College or the Committee on Academic Standing and Advising shall dismiss a student permanently from the College for academic reasons if that student returns from an academic suspension and then fails to earn a cumulative grade point average of 2.0 any semester after return, or who fails to meet other expectations as outlined by the Dean of the College for that student's readmission from suspension.

\oldbreak{VII-2}
%5.   
\item Academic Conduct Probation

The Committee on Academic Standing and Advising may place on Academic Conduct Probation any student who has been convicted of a conduct offense and for whom such probation has been recommended by the Community Court. Length of probation will be determined by the committee upon recommendation of Community Court.

%6.
\item College Rights

The College reserves its legal right to suspend or dismiss students for academic reasons. Faculty, administrators, and trustees shall not be held liable in cases of suspension or dismissal for academic reasons.
\end{enumerate}

\oldbreak{VII-3}

%Section 2. Course Grades
\subsubsection{Course Grades}\label{sub:CourseGrades}

%A.
\paragraph{The Grading Scale}

\begin{enumerate}[label=\itemLevelA,ref=\itemRefA]
%1.
\item Letter Grades
\begin{enumerate}[label=\itemLevelB,ref=\itemRefB]
\item In recording and reporting grades, the letters of the alphabet shall be used in such manner that A+ shall represent 97-100; A, 93-96; A-, 90-92; B+, 87-89; B, 83-86; B-, 80-82; C+, 77-79; C, 73-76; C-, 70-72; D+, 67-69; D, 63-66; D-, 60-62; F, 59 and below. 
\item The lowest passing grade shall be D-. 
\item At the discretion of the instructor, grades may be designated as Pass with Distinction (PD), Satisfactory (S), Fail (F), but such designation must be included in the catalog description of the course. In the case of internships, all internships must be graded Credit/Fail.

\item The grade of A indicates work of the highest quality; such work will generally be characterized not only by accuracy but also by excellence in such qualities as comprehensiveness, insight, and originality. The grade of B indicates work of good quality; such work will often show some of the qualities that characterize A work. The grade of C indicates work of average quality; such work will generally be reasonably accurate, buy may show only limited comprehensiveness, insight, and originality. The grade of D indicates work that is below average in quality but yet acceptable; such work may be unsatisfactory in certain respects, but will be satisfactory in others. The grade of F indicates work that is unsatisfactory.

\item For the computation of academic standing\modified{5/17/89}, a grade-point system is used. In this system A=4, B=3, C=2, D=1, and F=0, with an increase of 0.3 for each plus, and a decrease of 0.3 for each minus attached to a letter grade above F. The grade of PD has a numerical equivalent of 4 and should be assigned according to criteria analogous to those for A- or better.
\end{enumerate}
%2.
\item The Grade of Incomplete

\begin{enumerate}[label=\itemLevelB,ref=\itemRefB]
\item Under extraordinary circumstances \modified{5/17/89}an instructor may assign a grade of I (incomplete) if a student is unable to complete the work of a course on schedule but will be able to complete it at a later date without further class attendance. The extraordinary circumstances must be beyond the student's control (e.g., reasons of health or severe personal contingencies), and they must be documentable. The need for the Incomplete must have become apparent within the last three weeks of classes of a semester, and the student must have been passing the course at that time.

\item The student must file the incomplete grade \modified{4/9/96}request and contract form with the Registrar before leaving campus for the semester. Both the student and the instructor must sign the contract. If a contract is not submitted, the instructor may not assign an incomplete but shall assign the grade that the student would earn without completing the remaining work for the course. The incomplete grade request and contract form shall specify the requirements yet to be completed and the deadline for completion (no later than the end of the eighth week of the subsequent semester). The form shall also specify what the grade will be if the work is not completed.

\oldbreak{VII-4}

\item The Registrar shall send the instructor a reminder of the incomplete on the due date specified on the form. If the student has completed the work, the instructor shall duly submit the final grade to the Registrar within 5 days of the date specified on the form. If a student does not complete the work by the specified deadline, the instructor may change the ``I'' to the grade indicated on the form, or, if circumstances warrant it, the instructor may extend the deadline for the incomplete, but no later than the end of the semester after the incomplete was granted. If the Registrar does not receive either a grade or written notification of an extension by 5 days after the reminder, the Registrar will record the grade indicated on the form. If no grade is indicated, the Registrar shall record a grade of ``F*'' (to indicate an administrative assignment of a failing grade).

\item For any grades that remove grades of ``I'' submitted after grades are mailed to the student the Registrar shall note the date of the removal of the incomplete on the student's transcript. Until the Registrar substitutes a new grade, the ``I'' remains on the record.

\item Students with pending grades of ``I'' will not be allowed to participate in internships or off-campus study programs in the subsequent semester.
\end{enumerate}
%3. 
\item Administrative Assignment of a Failing Grade

If the Registrar does not receive a grade or an incomplete grade request and contract form in lieu of a grade, the Registrar may assign a grade of F* for the course. Such assignments may be made only with approval of the Dean of the College.
\end{enumerate}

%%%%%%%%%%%%%%%%%%%%%%%%%%%%%%%%% SAFE ABOVE

%B.
\paragraph{Determination of Grades}\label{par:DeterminationOfGrades}



\begin{enumerate}[label=\itemLevelA,ref=\itemRefA]
%1.
\item Criteria

 The final examination in a  course shall normally count one-third and the work up to the final examination two-thirds of the grade given a student at the close of a course.

%2.
\item Conspicuous Failure

In case of a conspicuous failure in a final examination, when the student, in the opinion of the instructor, fails to show anything like a satisfactory comprehension of the subject, the student may be marked as failing even though the numerical value of the class work would yield a grade of D-.

\item Report of Grades \label{item:ReportOfGrades}

\begin{enumerate}[label=\itemLevelB,ref=\itemRefB]

%a.
\item At the end of the first seven weeks \modified{12/9/08}of each semester, instructors shall report to the Registrar the grades of all new students, and grades of C- or below for any other student.

\oldbreak{VII-5}

%b.
\item  In each semester, final grades shall be submitted  to the Registrar within five days following the course examination. Grades for courses without a final examination are due within five days after the end of the examination period. Final grades for Seniors in the Spring Semester shall be reported to the Registrar at the time specified.

%c.
\item  The Registrar shall notify the student, academic advisors, and appropriate administrators of all grades reported, including Incomplete. 

%d.
\item  If an instructor \modified{4/9/96}wishes to change a recorded grade,  a written request for the change will be submitted to the Dean of the College. Except under extraordinary circumstances, as determined by the Dean, grades will be changed only through the eighth week of the subsequent semester and only because of the instructor's corrected calculation of the grade or because of the instructor's decision to consider ``lost'' work submitted by the student. Upon approval by the Dean, the change will be recorded by the Registrar. The Registrar will notify the student, in writing, within two weeks of the change in the recorded grade.
\end{enumerate}
\end{enumerate}

%Section 3. Examinations
\subsubsection{Examinations}

%A.
\paragraph{Final Examinations}

%1.
\begin{enumerate}[label=\alph*]
\item General Regulations

\begin{enumerate}[label=\itemLevelB,ref=\itemRefB]
%a.
\item Each instructor shall hold a three hour examination at the end of each course at the time indicated in the published schedule, except as noted in 2 below (Substitutions). No instructor is permitted to close a course of study or hold a final examination before the appointed time except that, at the discretion of the instructor, (a) student(s) may take the examination at a time other than that at which the rest of the students in the course are taking it.
%b.

%c.
\item  In courses in which papers are part of the work, but in which there is also a final examination or substitute therefore, instructors may not set a time for submitting papers any later than 4:30 p.m. on the day immediately following the last day of classes.
%d.
\item  Final examination books shall be returned to the students upon request. Examination books that are not returned will be kept on file by the instructor for a period of at least one year.
\end{enumerate}

%2.
\item Substitutions for Final Examinations

\begin{enumerate}[label=\itemLevelB,ref=\itemRefB]
\item An instructor may substitute in any course a form of evaluation other than examination, provided that notification of such intention shall be given to the Registrar, at the time specified. The Registrar's records of such substitutions may be reviewed by the Curriculum Committee.
\item Papers and projects assigned in lieu of a final examination will be due no later than the conclusion of the last scheduled examination.
\end{enumerate}

\oldbreak{VII-6}

%3.
\item Rules Governing Examinations

\begin{enumerate}[label=\itemLevelB,ref=\itemRefB]
\item Students are expected to govern themselves in examinations, as elsewhere, according to the rules and regulations of the Collegiate Association.
\item Should a student find it necessary to take an examination \textit{in absentia, }the student will be responsible for making necessary arrangements in consultation with the instructor. If the instructor requests, the student will specify the time and place the examination will be given and/or the name, address and telephone number of an individual who will administer the examination.
\end{enumerate}
\end{enumerate}

%B.
\paragraph{Comprehensive Evaluation}

\begin{enumerate}[label=\itemLevelA,ref=\itemRefA]
%1.
\item The Requirement

A comprehensive evaluation is required of each senior, except that a candidate in a Dual Degree Program may schedule the evaluation in other than the senior year. Some or all of the work upon which the evaluation is based will not carry course credit, but will be in addition to the normal course load.
%2.
\item Consultation with Students

In consultation with its  majors and with the approval of the Curriculum Committee, each department or programmatic area shall  set forth the means by which students then in the junior class will meet the goal of comprehensive evaluation.

%3.
\item Report of Grades

Each department or programmatic area shall submit to the Registrar a grade of Distinction, Satisfactory, or Fail for each student. The faculty will report to each student the result of the comprehensive evaluation.

\end{enumerate}

%Section 4.Records
\subsubsection{Records}

%A.
\paragraph{Instructor's Records}

Each instructor shall keep a record of graded work of the students in each of their courses for at least three years.

%B.
\paragraph{Registrar's Records}

\begin{enumerate}[label=\itemLevelA,ref=\itemRefA]
%1.
\item Course Records

\begin{enumerate}[label=\itemLevelB,ref=\itemRefB]

\item The Registrar shall keep a file of course syllabi  and final examinations for all courses for at least three years. Each instructor is responsible for submitting this material to the Registrar each semester.

%b.
\item The Registrar\modified{5/12/98}  shall keep a written record of all independent study courses a student takes in the student's file.

\end{enumerate}
\oldbreak{VII-7}

%2.
\item Student Records

\begin{enumerate}[label=\itemLevelB,ref=\itemRefB]

\item Under the provisions of the federal Family Educational Rights and Privacy Act of 1974, a student may review College records, files, and data directly related to themselves except: 

\begin{enumerate}[label=\alph*)]

\item medical and psychiatric records; 

\item confidential recommendations submitted before January 1, 1975;

%3. 
\item records to which the right of access has been waived; 

%4. 
\item and parents' financial records.
\end{enumerate}

\item Any student may seek correction or deletion of records which the student feels are inaccurate, misleading, or in violation of their rights. The student will initiate such action by consulting the Dean of the College for records concerning academic matters or the Dean of Students for records concerning non-academic matters.

\item Information from the records of a student who is a dependent of their parents for Internal Revenue Service purposes may be released to the parents without the student's consent. Unless a student has filed notification of financial independence and supporting evidence of that independence with the Dean of Students by the first day of classes of each academic year, they will be assumed to be dependent.

%c. 
\item College instructors and officials of Wells who have a legitimate educational reason for reviewing a student's records will have access to those records.

%d.
\item The Privacy act allows Wells College to make public, without consent, the following information about individual students:
\begin{enumerate}[label=\alph*)]
%1.
\item name, class and year of graduation;
%2.
\item home address and telephone number;
%3.
\item college address and telephone number;
%4.
\item major field;
%5.
\item date and place of birth;
%6.
\item dates of attendance at Wells;
%7.
\item degrees, honors and awards received;
%8.
\item height and weight of athletes;
%9.
\item participation in officially recognized activities;
%10.
\item previous educational institution most recently attended.
\end{enumerate}

Any student who wishes to limit the release of any of the above information must inform the Dean of Students in writing no later than the first day of classes each academic year.

\item A student's permanent record will include a record of all courses and grades, honors received, and participation in officially recognized activities. All other materials relating to the students's academic performance will be destroyed within one year following graduation.

\end{enumerate}
\end{enumerate}
\oldbreak{VII-8}

%ARTICLE VIII. ADMISSION OF STUDENTS TO ADVANCED STANDING
\subsection{Admission of Students to Advanced Standing}\label{sec:AdmissionOfStudentsToAdvancedStanding}
%Section 1. Eligibility
\subsubsection{Eligibility}

No new student may be admitted to the Senior Class. A candidate for advanced standing must, in general, have met the requirements for admission to the Freshman Class and present academic records which are distinctly creditable.

%Section 2. Assignment of Credit
\subsubsection{Assignment of Credit}\label{sub:AssignmentOfCredit}

\paragraph{Liberal Arts Credit}

%A.
\begin{enumerate}[label=\itemLevelA,ref=\itemRefA]
\item A student holding an A.A. or A.S. degree in Liberal Arts from an accredited institution with whom Wells has an articulation agreement normally will receive two years of credit towards the Wells degree without further review of  courses, if  admitted to Wells.

%B.
\item A student will normally receive full credit for work in courses in the Liberal Arts \modified{9/14/99} at an accredited institution if a grade of C- or better is recorded. Students entering Wells College under an articulation agreement will be granted credit according to the agreement. The Dean of  the College or the Registrar will evaluate courses to determine their correspondence to Wells courses.

%C.
\item Decisions about credit for Liberal Arts courses that do not correspond substantially to Wells courses will be made by the Dean of the College after consultation with a representative of the department most closely allied with the course in question.
\end{enumerate}

%D.
\paragraph{Non-Liberal Arts Credit}
A student may receive a maximum of 12 Wells semester hours for work in courses outside the Liberal Arts at an accredited institution if a grade of D- or better is recorded. Decisions about the suitability of these courses for Wells credit will be made by the Dean of the College or the Registrar after consultation with a representative of the department most closely allied with the course in question.

%E.
\paragraph{Placement in Advanced Courses}
Placement in advanced courses will be at the discretion of the department involved with the understanding that a student can not take for credit a Wells course that substantially duplicates a transfer course for which the student has credit. Applicability of any transfer courses to requirements for the major will be at the discretion of the department.

\oldbreak{VIII - 1}

%ARTICLE IX. ABSENCES
\subsection{Absences}

%Section 1. General RegulationsMajor Revisions: 5/14/02
\subsubsection{General Regulations}\modified{5/14/02}

It is the general policy of the College that class attendance is expected, but instructors have the right and the obligation to set their own policies regarding absences.

%Section 2. Work Missed Because of Absence
\subsubsection{Work Missed Because of Absence}

%A.
\paragraph{Responsibility}
It is the responsibility of a student who is absent from classes because of illness or family emergency will arrange to make up missed work. The Dean of Students will notify faculty about students who have been admitted to a healthcare facility or who must unexpectedly leave the campus for reasons of illness or family emergency, and will request the cooperation of the faculty.

%B.
\paragraph{Deadlines}
Where permission to complete missed or late work has been granted by the faculty member, the faculty member will set a deadline for the completion of the work and will determine whether or not a penalty is to be exacted for lateness. Once the deadline has been set no further extensions save for exceptional circumstances will be allowed and the student alone will be responsible for seeing to it that all obligations are fulfilled as stipulated. In no case may deadlines be set beyond the times specified in the Manual for turning in grades to the Registrar (see \cref{art:RegulationsConcerningStudentStatus}, \cref{sec:AcademicRegulations}.\labelcref{sub:CourseGrades}.\labelcref{par:DeterminationOfGrades}.\labelcref{item:ReportOfGrades}).

%C.
\paragraph{Honor Suspension}
A student suspended for violation of the honor code shall not receive credit for any enrolled courses during the semester for which the suspension is imposed.

%Section 3. Leaves of Absence 
\subsubsection{Leaves of Absence}

%A.
\paragraph{Leave of Absence} \modified{5/9/95}

\begin{enumerate}[label=\alph*)]
\item A leave of absence is granted to a student who must be absent during the semester but anticipates returning to complete the course work. A leave of absence for medical or other personal reasons may be granted by the Dean of Students for a maximum of 15 days; only one such leave may be granted during a 12-month period. When a student initiates the leave of absence, the Dean of Students Office will notify the academic advisor and the faculty in whose courses the student is registered.
\item A student who does not return from the approved leave of absence will be withdrawn from the College as of the last documentable date of attendance in class. (See \cref{sub:WithdrawlFromTheCollege})
\end{enumerate}

\oldbreak{IX-1}

%B.
\paragraph{Extended Leave of Absence}

\begin{enumerate}[label=\alph*)]
\item Extended Leave of Absence: An extended leave of absence is granted to a student who must be absent for more than 15 days for medical or other personal reasons. An extended leave shall not exceed two semesters and shall be granted by the Dean of Students. 
\item When a student is on extended leave of absence, Wells College cannot certify enrollment for  any semester during which the student is on such leave.
\end{enumerate}

%Section 4.Approved Off-Campus Study
\subsubsection{Approved Off-Campus Study}

A student studying off-campus is participating in one of the following options.

%a.
\paragraph{Affiliated Programs}
Affiliated Programs -- administered by Wells College. Students are registered full-time through the college; thus, Wells College can certify enrollment.

%b.
\paragraph{Field Experience}
Field Experience -- an approved semester-long internship or independent study experience. Students register for such an experience through the college; thus, Wells College can certify enrollment according to the number of semester hours for which the student is registered.

%c.
\paragraph{Non-Affiliated Programs}
Non-Affiliated Programs -- administered by other institutions. Students are registered at the offering institution and not at Wells College; thus Wells College cannot certify enrollment.

%Section 5.Withdrawal from the College
\subsubsection{Withdrawal from the College}\label{sub:WithdrawlFromTheCollege}

A student withdraws when intending to not return to Wells College.

%A.
\paragraph{Procedure}
\begin{enumerate}[label=\alph*)]
\item A student wishing to withdraw from the College before the completion of a semester is required to comply with the withdrawal procedure. This process is initiated with the Dean of Students and requires completion of the withdrawal form. Students may not withdraw after the last day of classes. When a student has officially withdrawn, the Dean of Students Office will notify the academic advisor and the faculty in whose courses the student was registered.

\item A student who is completing the current semester but is not intending to return to Wells for the next semester should also complete a withdrawal form and indicate the reasons for leaving the College.
\end{enumerate}
%B.
\paragraph{Grading Policy}

When a student withdraws from the College, the transcript will read as follows:

\begin{enumerate}[label=\alph*)]

%a.
\item Before the first day of classes, no record is entered.

%b.
\item From the first class day through the official withdrawal period, the registrar will assign a ``W'' for each course.

\oldbreak{IX-2}

\item After the official withdrawal period, students who withdraw from the College shall \modified{2/12/02 } receive grades W if the student was passing the course at the time of withdrawal, or WF if the student was failing the course at the time of withdrawal, except in cases in which--by the judgment of the Dean of the College--the withdrawal was in effect required for medical or other grave personal reasons. In such cases, the Registrar shall record W.

\end{enumerate}

%C.
\paragraph{Readmission}
\begin{enumerate}[label=\alph*)]
\item A student in good standing who withdraws or takes an extended leave of absence from the College will be readmitted by the Dean of Students. A physician's report may be required for readmission of a student granted an extended leave for medical reasons.

\item A student who is suspended from the College for academic reasons may be readmitted by the Dean of the College.

\item A student who is suspended from the College for non-academic reasons may be readmitted by the Dean of Students.
\end{enumerate}

\oldbreak{IX-3}

%ARTICLE X. NON-TRADITIONAL STUDENTS
\subsection{Non-Traditional Students}\label{sec:NonTraditionalStudents}


%%Section 1. Women in Lifelong Learning[5/9/95]
%\subsubsection{Women in Lifelong Learning}\modified{5/9/95}
%
%%A.
%\paragraph{Definition}
%
%\editRemark{Sanity Check}{Women who have reached the age of 24 are designated as WILL students. A WILL student may be full- or part-time.}
%
%%B.
%\paragraph{Application Procedure}
%
%\editRemark{Sanity Check}{Application for admission follows the same procedure as for traditional age students.}
%
%%C.
%\paragraph{Academic Advising}
%
%\editRemark{Sanity Check}{The Director of Academic Advising, or a designated faculty advisor shall serve as the academic advisor to WILL students who have not declared a major. Students who have declared a major will be assigned an advisor in the appropriate department.}
%
%%D.
%\paragraph{Credit}
%
%\editRemark{Sanity Check}{A WILL student may be granted up to 60 semester hours of approved credit following the criteria as listed in \cref{art:RegulationsConcerningStudentStatus}, \cref{sec:RegulationOfStudies}.\labelcref{sub:ChoiceOfStudies}.\labelcref{par:AcquiringCredit}.\labelcref{iitem:incomingCredit}.}

%Section 2.Special Students
\subsubsection{Special Students}\label{sub:SpecialStudents}


%A.
\paragraph{Definition}

Individuals who wish to enroll in courses on a course-by-course or semester-by-semester basis may do so for up to 32 credit hours as special students. Upon completion of 32 semester hours, special students must either request permission from the Dean of the College to continue,  may apply to the College for admission into a degree program. Special students may enroll in classes on a space available basis only.

%B.
\paragraph{Categories of Special Students}\label{par:CategoriesOfSpecialStudents}

\begin{enumerate}[label=\itemLevelA,ref=\itemRefA]

%1.
\item Accelerated High School Student: A high school junior or senior may take up to two classes a semester.
%2.
\item  Visiting/Exchange Student: A student in good standing from another college or university may attend Wells full time as a visiting student for a semester or a year.
%3.
\item  Post-Baccalaureate Student: A student who has already received a bachelor's degree may take up to 32 hours. The student must have a 2.7 GPA.
%4.
\item Senior Citizen: Anyone over the age of 55 may audit one class a semester free of charge.

\oldbreak{X-1}

%5.
\item \label{item:employeeClasses} Wells College Employee: A Wells College employee is eligible to take up to 32 semester hours with permission from the appropriate senior staff. Employees may also seek admission into a degree program.

%6.
\item \label{item:spouseClasses} Child/Spouse of a Wells College Employee: The child or spouse of an employee is eligible for full-time enrollment up to 32 semester hours, after which time they may apply for admission to the College in a degree program.

%7.
\item Cornell University Student: A Cornell student is eligible to take one class a year with permission from Cornell University.
\end{enumerate}

%B.
\paragraph{Application Procedure}

Special students submit the Special Student Application for admission to the Admissions Office.

%C.
\paragraph{Academic Advising}

The Director of Academic Advising shall serve as the academic advisor to special students.

\oldbreak{X-2}

%ARTICLE XI. THE LIBRARY
\subsection{The Library}

Rules and regulations for the withdrawal and return of books, and other regulations pertaining to the use of the Library shall be decided by the Librarian with the approval of the Dean's Council.

%ARTICLE IV. AMENDMENT
\section{AMENDMENT}\label{art:Amendment}

\subsection{This Manual}
The \facman~ may be amended by the Faculty at any meeting, provided notice of the proposed amendment has been given at the previous meeting.

\subsection{Regulations Concerning Faculty Status}
Amendment of the rules in \cref{art:RegulationsConcerningFacultyStatus} require approval of the Faculty and the Board of Trustees (see \cref{art:RegulationsConcerningFacultyStatus}, \cref{sec:Revision}).

\end{document}

